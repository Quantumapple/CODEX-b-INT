\section{Exotic structures in $\Bp \to \jpsi \phi \Kp$}

\frame{
  \frametitle{The $X(4140)$ chronology}
   \begin{columns}
     \column{0.33\textwidth}
       \centering
       \includegraphics[width=1.4in]{figs/jpsiphik/jpsiphik_7.pdf}
     \column{0.33\textwidth}
       \centering
       {\scriptsize PRD 85, 091103R}\\
       \includegraphics[width=1.4in]{figs/jpsiphik/jpsiphik_8.pdf}
     \column{0.33\textwidth}
       \centering
       {\scriptsize PLB 734, 261}
       \includegraphics[width=1.4in]{figs/jpsiphik/jpsiphik_9.pdf}
   \end{columns}


   \begin{columns}
     \column{0.33\textwidth}
       \centering
       \includegraphics[width=1.4in]{figs/jpsiphik/jpsiphik_10.pdf}
     \column{0.66\textwidth}
     \begin{itemize}
     	\item \textcolor{red}{$X(4140)$}: some \textcolor{red}{disagreements} over the years from hadron colliders.
     	\vspace{0.2cm} 
        \item Also some results from the $B$-factories. Not too significant, but neither in contradiction with hadron colliders.
     \end{itemize}
   \end{columns}
}


\frame{
  \frametitle{$X$ states in 3/fb LHCb (NEW!) {\scriptsize LHCb-Paper-2016-018}}
  
   \begin{columns}
     \column{0.5\textwidth}
       \centering
       \includegraphics[width=2.3in]{figs/jpsiphik/jpsiphik_11.pdf}
     \column{0.5\textwidth}
     \begin{itemize}
        \item Selection similar to 0.37/fb \href{https://arxiv.org/abs/1202.5087}{analysis}, with some re-optimization. 
        \vspace{0.2cm}
        \item \textcolor{red}{$N_{\rm sig}= 4289\pm 151$}, background fraction $23\pm6\%$ inside signal band.
        \vspace{0.2cm}
        \item Largest world dataset for this mode.
     \end{itemize}
   \end{columns}
     \begin{itemize}
        \item First amplitude analysis in 6d: \textcolor{red}{$m_{\phi K}$} and \textcolor{red}{5 angular} variables%, similar to $P_c/Z(4430)$ analyses.
        \vspace{0.2cm}
        \item Three contributing amplitudes that can interfere:
        \begin{itemize}
        \normalsize
                   \item $\Bp \to \jpsi \textcolor{blue}{K^{\ast +} (\to \phi K^+)}$ (poorly understood $K^\ast$ states)
           \item $\Bp \to \textcolor{red}{X (\to \jpsi \phi)} \Kp$ (``exotic'' $X$ states) 
 
           \item $\Bp \to \textcolor{magenta}{Z^+_c (\to \jpsi K^+)}\phi$ (``exotic'' $Z$ states)
        \end{itemize}
     \end{itemize}
}


\frame{
  \frametitle{Fit results {\scriptsize LHCb-Paper-2016-018}}

   \begin{columns}
     \column{0.5\textwidth}
       \centering
       \includegraphics[width=2.1in]{figs/jpsiphik/jpsiphik_5.pdf}
     \column{0.5\textwidth}
     \vspace{-0.2cm}
     \begin{itemize}
        \item No obvious peaking structure in $m_{\phi K}$, but rich $K^\ast$ structure underneath.
        \vspace{0.2cm}
        \item Adding $Z^+\to \jpsi K^+$ doesn't have much effect. Only adding $X$ states improves fit. 
     \end{itemize}
     \vspace{0.05cm}
   \end{columns}


   \begin{columns}
     \column{0.5\textwidth}
       \centering
       \includegraphics[width=2.1in]{figs/jpsiphik/jpsiphik_1.pdf}
     \column{0.5\textwidth}
       \centering
       \includegraphics[width=2.1in]{figs/jpsiphik/jpsiphik_2.pdf}
   \end{columns}
}


\frame{
  \frametitle{Fit results for $K^\ast\to \phi K$ {\scriptsize LHCb-Paper-2016-018}}
   \vspace{-0.2cm}
   \begin{columns}
     \column{0.5\textwidth}
       \centering
       \includegraphics[width=2.6in]{figs/jpsiphik/jpsiphik_6.pdf}
       \vspace{-0.3cm}
       \begin{itemize}
         \item Good agreement with both theory (\href{http://journals.aps.org/prd/abstract/10.1103/PhysRevD.32.189}{Godfrey-Isgur}) and previous experiments.
       \end{itemize}

     \column{0.5\textwidth}
       \begin{itemize}
         \item \textcolor{red}{$K^\ast$}: within kinematic limits, \textcolor{red}{red circles} are the \textcolor{red}{LHCb} fit results.
         \vspace{0.1cm}
         \item \textcolor{red}{$1^+$}: NR + 1793(1900) + 1968(1930). FF is \textcolor{red}{$42\%$}.
         \vspace{0.1cm}
         \item \textcolor{red}{$2^-$}: 1777(1770) + 1853(1820). FF is $11\%$.
         \vspace{0.1cm}
         \item \textcolor{red}{$1^-$}: $1717(1680)$. FF is $6.7\%$. 
         \vspace{0.1cm}
         \item \textcolor{red}{$2^+$}: 2073(1980). FF is $2.9\%$.
         \vspace{0.1cm}
         \item \textcolor{red}{$0^-$}: consistent with unconfirmed 1874(1830). FF is $2.6\%$. 
                  \vspace{0.1cm}
         %\vspace{0.1cm}
         %\item FF for: $X(0^+)$ is $27.6\%$, all $X(1^+)$ is $16.0\%$ 
       \end{itemize}
   \end{columns}
}


\frame{
  \frametitle{Fit results for $X\to \jpsi \phi$ {\scriptsize LHCb-Paper-2016-018}}
   \vspace{-0.4cm}
   \begin{itemize}
     \item \textcolor{red}{$X(4140)$} mass consistent with previous measurements. \textcolor{red}{Width} larger.
     \vspace{0.1cm}
      \item \textcolor{red}{$X(4274)$} mass/width consistent with unpub. \href{http://xxx.lanl.gov/abs/1101.6058}{CDF results}.
      \vspace{0.2cm}
      \item $X(4140)$ and $X(4274)$: \textcolor{red}{$J^{PC}= 1^{++}$} determined for the 1st time. $X(4140)$ can be a possible $D^\pm_s D_s^{\ast \mp}$ ``cusp'' (rescattering).
     \vspace{0.1cm}
     \item \textcolor{blue}{High $m_{\jpsi \phi}$} mass region probed for the 1st time. Dominated by \textcolor{blue}{$0^-$}. \textcolor{blue}{NR} + new \textcolor{blue}{$X(4500)$} and \textcolor{blue}{$X(4700)$} resonances.
   \end{itemize}
    %\vspace{0.2cm}
  \small
  \begin{table}
  \begin{center}
  \begin{tabular}{c|c|c|c|c|c}
    State & $J^{PC}$ & signif. & Mass & Width & fit frac. \\\hline
    \textcolor{red}{$X(4140)$} & \textcolor{red}{$1^{++}$} & $8.4\sigma$ & $4165\pm 4.5^{+4.6}_{-2.8}$ & $83\pm 21^{+21}_{-14}$ & $13.0 \pm 3.2^{+4.8}_{-2.0}$ \\ \hline
    \textcolor{red}{$X(4274)$} & \textcolor{red}{$1^{++}$} & $6.0\sigma$ & $4273.3 \pm 8.3^{+17.2}_{-3.6}$ & $56\pm 11^{+8}_{-11}$ & $7.1 \pm 2.5^{+3.5}_{-2.4}$ \\ \hline
    \textcolor{blue}{$X(4500)$} & \textcolor{blue}{$0^{++}$} & $6.1\sigma$ & $4506 \pm 11^{+12}_{-15}$ & $92\pm 21^{+21}_{-20}$ & $6.6 \pm 2.4^{+2.5}_{-2.3}$ \\ \hline
    \textcolor{blue}{$X(4700)$} & \textcolor{blue}{$0^{++}$} & $6.1\sigma$ & $4704 \pm 10^{+14}_{-24}$ & $120\pm 31^{+42}_{-33}$ & $12 \pm 5^{+9}_{-5}$ \\ \hline
  \end{tabular}
  \end{center}
  \end{table}

}


\frame{
	\frametitle{Interpretations already...}
	       \centering
	       \includegraphics[width=5in]{figs/jpsiphik/theory_arxiv.pdf}
}	