%\documentclass[handout]{beamer}
\documentclass[]{beamer}
\mode<presentation>
\usepackage{beamerthemesplit}
\usepackage{subfigure}
\usepackage{slashed}
\usepackage{color}
%\usepackage{pdffig}
\usepackage{pgf}
\usepackage{amsmath}
\usepackage{graphics}
\usepackage{graphicx}
\usepackage{fancybox}
\usepackage{subfigure}
\usepackage{animate}
\usepackage{pdfanim}
\usepackage{multirow}
\usepackage{sidecap} 
\usepackage{psfrag}
\usepackage{rotating}
\usepackage{amsmath}
%\usepackage{fcolorbox}
\usepackage{soul}
\usepackage{xcolor,cancel}
\usepackage{colortbl}
\usepackage[normalem]{ulem}
\newcommand\hcancel[2][black]{\setbox0=\hbox{$#2$}%
\rlap{\raisebox{.45\ht0}{\textcolor{#1}{\rule{\wd0}{1pt}}}}#2} 

\newcommand\Ccancel[2][red]{\renewcommand\CancelColor{\color{#1}}\cancel{#2}}

\usepackage{empheq}
%\newcommand{\boxedeq}[1]{\begin{empheq}[box={\fboxsep=6pt\fbox}]{align*}#1\end{empheq}}
%\newcommand{\coloredeq}[2]{\begin{empheq}[box=\colorbox{#1}]{align*}#2\end{empheq}}
%\newcommand{\highlight}[2]{\colorbox{#1}{$\displaystyle #2$}}


\usepackage[english]{babel}
\usepackage[T1]{fontenc}
\usepackage[utf8]{inputenc}
\usepackage{color,soul}
\makeatletter
\newcommand\SoulColor{%
  \let\set@color\beamerorig@set@color
  \let\reset@color\beamerorig@reset@color}
\makeatother
\setstcolor{red}
\setul{}{1.5pt} %Use this to change weight of underline/strikethrough

\newcommand{\displayst}[1]{\textrm{\SoulColor\Ccancel{$#1$}}}



\makeatletter
\def\UL@putbox{\ifx\UL@start\@empty \else % not inner
  \vrule\@width\z@ \LA@penalty\@M
  {\UL@skip\wd\UL@box \UL@leaders \kern-\UL@skip}%
    \phantom{\box\UL@box}%
  \fi}
\makeatother

% Variable width alert block
\newenvironment<>{varalertblock}[2][0.9\textwidth]{%
    \setlength{\textwidth}{#1}%
    \setlength{\linewidth}{\textwidth}%
  \begin{actionenv}#3%
    \def\insertblocktitle{#2}%
    \par%
    \setbeamercolor{local structure}{parent=alerted text}%
    \usebeamertemplate{block alerted begin}}
  {\par%
  \usebeamertemplate{block alerted end}%
    \end{actionenv}}



\definecolor{ah1col}{rgb}{0.84, 0.84, 0.84}

%\usetheme{AnnArbor}
%\usetheme{CambridgeUS}
\usetheme{Madrid}


%\setbeamercolor{palette primary}{fg=yellow, bg=black}
%\setbeamercolor{palette secondary}{fg=black, bg=white}
%\setbeamercolor{palette tertiary}{fg=yellow, bg=green}


% Uncomment these to make handouts
%\usepackage{pgfpages}
%\pgfpagesuselayout{4 on 1}[a4paper, border shrink=5mm]
%\usetheme{default}

%Font and colour commands
%\usefonttheme[onlylarge]{structuresmallcapsserif}
%\usefonttheme[onlysmall]{structurebold}
%\setbeamerfont{title}{shape=\itshape,family=\rmfamily}
%%\setbeamercolor{title}{fg=red!80!black, bg=red!20!white}
%\setbeamercolor{title}{fg=yellow, bg=black}
%\setbeamercolor{block body}{bg=black, fg=white}
%\usepackage{palatino}
%\usepackage{charter}
%\sffamily
%bf

%Uncomment this to make unshown listings slightly
%visible
%\setbeamercovered{transparent}


%\title[{\tt DPF'13, UCSC}]{Exclusive $B\to X_{\{c,u\}} \ell^- \bar{\nu}$ angular analyses \\with hadronic tagging from $\babar$}
%\title[]{Right-handed weak charged currents via exclusive \\ $\Bbar\to X_{\{c,u\}} \ell^- \bar{\nu}_\ell$ angular analyses \\with hadronic tagging from $\babar$}
%\title[{\tt S, P, D waves}]{$B \to X \ell \ell$ }
%\title[Vetoing using DTF]{ Some ideas for optimization of peaking component vetoing for $\overline{B}^0 \to \psi K^-\pip $}
%\title[Moments method]{\textcolor{yellow}{Doing an angular fit without doing an angular fit:\\ the moments technique}}
%\title[{\tt EvtXLL.cpp}]{\textcolor{yellow}{{\tt XLL}: an optimized MC generator for $b \to h h\mu\mu$}}
%\title[$\psi(2S) \to J/\psi (\to \mup \mu^-) \pi \pi$]{$B_d \to \psi' K \pi$ and $B_s \to \psi' K K$ angular analyses\\ using the $\psi' \to \psi (\to \mup \mu^-) \pi \pi$ mode}
%\title[$\psi(2S) \to J/\psi (\to \mup \mu^-) \pi \pi$]{$B_d \to \psi(2S) K \pi$ and $B_s \to \psi(2S) K K$ angular analyses\\ using the $\psi(2S) \to J/\psi (\to \mup \mu^-) \pi \pi$ mode}
%\title[$\psi(2S) \to J/\psi (\to \mup \mu^-) \pi \pi$]{A proposal to break the two-fold ambiguity in \\ $B_d \to \psi(2S) K \pi$ and $B_s \to \psi(2S) K K$ amplitudes \\ using $\psi(2S) \to J/\psi (\to \mup \mu^-) \pi \pi$ }

\title[CODEX-b:backgrounds \& simulations]{\textcolor{yellow}{Background measurements and simulations \\ for CODEX-b}}

\date[28 Aug, 2018]{}
%\date[Nov. $26^{th}$, 2013]{{\tt Physik-Institut, UZH, Nov. $26^{th}$, 2013}}}

%\subtitle[]{(BAD \# 2566)}

\author[Jongho Lee]{{\Large Jongho Lee}\\ {\footnotesize \vspace{0.25cm} LHCb summer student presentation \\ \vspace{0.25cm} Supervisors: Victor Coco, Biplab Dey \\ \vspace{0.25cm} 28th Aug, 2018, CERN} \\ \vspace{0.5cm} \includegraphics[width=3cm]{pictures/LHCbLogo.jpg} }


%\author[Biplab Dey]{\vspace{-0.5cm}{\Large \underline{Biplab~Dey}, Nico Serra, Bill Dunwoodie}} 
%\author[Biplab Dey]{\vspace{-0.5cm}{\Large \underline{Biplab~Dey}}\vspace{0.7cm} \\{\small {\tt Physik-Institut, Universit{\"a}t Z{\"u}rich}} \vspace{0.1cm} \\ {\scriptsize Nov. $26^{th}$, 2013}
%\author[Biplab Dey]{{\Large \underline{Biplab~Dey}, Owen Long and Bill Gary \\ (w/ help from Bill Dunwoodie)}

%\institute[\babar]{\normalsize March 2018 \babar Jamboree}
%\institute[]{
%\vspace{1cm}

%{\flushbottom
% \includegraphics[width=1.9in]{figs/logos/uni_zurich_pi_logo.png} %\hspace{1.1cm}
 %\vspace{-1cm}\includegraphics[width=1.2in]{figs/logos/SLAC_Logo.jpg}\hspace{1.6cm} 
%}
%}


\newcommand {\bmdpilnu} {\Bb \to D^\ast (\to D \pi) \ell^- \overline{\nu}_\ell}
\newcommand {\bpdpilnu} {B \to \overline{D}^\ast (\to \overline{D} \overline{\pi}) \ellp \nu_\ell}

\newcommand {\bdpilnu} {\overline{B} \to D^\ast (\to D \pi) \ell^- \overline{\nu}_\ell}
\newcommand {\epsR} {\epsilon_R}
\newcommand {\bdstlnu} {\overline{B} \to \Dstar \ell^- \overline{\nu}_\ell}
\newcommand {\bprlnu} {B^+ \to \rhoz \ell^+ \nu_\ell}
\newcommand {\bpipilnu} {\Bm \to \pip \pim \ell^- \overline{\nu}_\ell}
\newcommand {\brlnu} {\overline{B} \to \rhoz \ell^- \overline{\nu}_\ell}
\newcommand {\barnuell} {\overline{\nu}_\ell}
\newcommand {\ctn} {\cos \theta_\nu}
\newcommand {\ctv} {\cos \theta_V}
\newcommand {\thetav} {\theta_V}
\def\thetal     {\theta_\ell}
\def\thetaV     {\theta_V}
\def\GF   {G_F} %Fermi constant
\def\BB      {\ensuremath{B\overline{B}}\xspace}

% Huge boldface
\def\hbabar{\mbox{{\huge\bf\sl B}\hspace{-0.1em}{\LARGE\bf\sl A}\hspace{-0.03em}{\huge\bf\sl B}\hspace{-0.1em}{\LARGE\bf\sl A\hspace{-0.03em}R}}\;}
% LARGE
\def\Lbabar{\mbox{{\LARGE\sl B}\hspace{-0.15em}{\Large\sl A}\hspace{-0.07em}{\LARGE\sl B}\hspace{-0.15em}{\Large\sl A\hspace{-0.02em}R}}}
% Large
\def\lbabar{\mbox{{\large\sl B}\hspace{-0.4em} {\normalsize\sl A}\hspace{-0.03em}{\large\sl B}\hspace{-0.4em} {\normalsize\sl A\hspace{-0.02em}R}}}
% normal size
%\def\babar{\mbox{\sl B\hspace{-0.4em} {\small\sl A}\hspace{-0.37em} \sl B\hspace{-0.4em} {\small\sl A\hspace{-0.02em}R}}}
% replace normalsize with scalable version       dbm 7/4/00
\usepackage{relsize}
\def\babar{\mbox{\slshape B\kern-0.1em{\smaller A}\kern-0.1em
    B\kern-0.1em{\smaller A\kern-0.2em R}}\;}


\newcommand {\ups} {\Upsilon(4S)}
\newcommand {\eex} {E_{\mbox{\scriptsize ex}}}
\newcommand {\eext} {E_{\mbox{\scriptsize extra}}}
\newcommand {\btag} {B_{\mbox{\scriptsize tag}}}
\newcommand {\bsig} {B_{\mbox{\scriptsize sig}}}
\newcommand {\bbsig} {\overline{B}_{\mbox{\scriptsize sig}}}
\def\DeltaE     {\mbox{$\Delta E$}\xspace}
\newcommand {\breco} {B_{\mbox{\scriptsize reco}}}
\def\mes        {\mbox{$m_{\rm ES}$}\xspace}
\newcommand {\bsigb} {\overline{B}_{\mbox{\scriptsize sig}}}
\newcommand {\pmiss} {\vec{p}_{\mbox{\scriptsize miss}} }




\newcommand {\img} {\, Im}
\newcommand {\rel} {\, Re}

\newcommand {\vcb} {V_{cb}}





\input{rgb}
\begin{document}


\begin{frame}
  \titlepage
\end{frame}

% outline page
%\begin{frame}
%  \frametitle{Outline}
%  \tableofcontents
%\end{frame}

%\AtBeginSection[]
%\AtBeginSubsection[]


%\AtBeginSection[]{
%\addtocounter{framenumber}{-1}
%  \begin{frame}
%    \frametitle{Outline}
%    \tableofcontents[currentsection]
%  \end{frame}
%}

\section{Introduction}


%\frame{
%  \frametitle{Towards the SM - I}
%    \vspace{-0.2cm}
%    \begin{itemize}
%      \item Birth of \textcolor{red}{quantum mechanics} in the 1900's 
%    \end{itemize}
%    \vspace{0.3cm}
%    \centering
%    \includegraphics[width=3.6in]{figs/intro/sm0.pdf}
%    \vspace{0.5cm}
%    \begin{itemize}
%      \item Gave rise to \textcolor{red}{Quantum Field Theory}
%    \end{itemize}
%    \vspace{0.5cm}

%  \begin{columns}
%    \column{0.33\textwidth}
%    \begin{itemize}
%      \item Particles $\leftrightarrow$ fields 
%    \end{itemize}
%    \column{0.33\textwidth}
%    \begin{itemize}
%      \item Anti-matter
%    \end{itemize}
%    \column{0.33\textwidth}
%    \begin{itemize}
%      \item Special relativity
%    \end{itemize}
%  \end{columns}
%    \vspace{0.3cm}


%  \begin{columns}
%    \column{0.33\textwidth}
%    \centering
%    \includegraphics[width=1.2in]{figs/intro/sm_feynmandiag.jpg}
%    \column{0.33\textwidth}
%    \centering
%    \includegraphics[width=1in]{figs/intro/ep_em.jpg}
%    \column{0.33\textwidth}
%    \centering
%    {\large $E = m c^2$}
%  \end{columns}
%
%}

%\frame{
%  \frametitle{Towards the SM - II}
%    \vspace{-0.5cm}
%    \begin{itemize}
%      \item 1950's \textcolor{DarkGreen}{Quantum Electrodynamics} (renormalization)\\ 
%    \end{itemize}
%    \centering
%    \includegraphics[width=1.8in]{figs/intro/sm1.pdf}
%
%    \begin{itemize}
%      \item 1960's \textcolor{blue}{Electroweak theory} (Higgs, EW unification)\\ 
%    \end{itemize}
%    \centering
%    \includegraphics[width=1.8in]{figs/intro/sm2.pdf}


%    \begin{itemize}
%      \item 1970's \textcolor{red}{Quantum Chromodynamics} (confinement, asymptotic freedom)\\ 
%    \end{itemize}
%    \centering
%    \includegraphics[width=1.8in]{figs/intro/sm3.pdf}

%}


\frame{
  \frametitle{The Standard Model}
    \vspace{-0.3cm}

  \begin{columns}
    \column{0.5\textwidth}
      \begin{itemize}
      \item A remarkably successful theory!\\ \phantom{bla}
      \end{itemize}
    \column{0.5\textwidth}
       
   \begin{itemize}
      \item Fine structure constant\\
   $\alpha = 0.0072973525698(24)$
   \end{itemize}
  \end{columns}
 \vspace{0.4cm}
    \centering
    \includegraphics[width=3.6in]{figs/intro/sm.pdf}


   \begin{itemize}
      \item Has withstood \textcolor{red}{experimental tests} for over 40 years...
   \end{itemize}
}


\frame{
  \frametitle{Yet, major difficulties...}
   \begin{itemize}
      \item How does \textcolor{red}{gravity} fit in?
      \vspace{0.4cm}
      \item \textcolor{red}{Matter-antimatter asymmetry}, \textcolor{red}{neutrino oscillations}, origin of mass.
      \vspace{0.4cm}
      \item Ordinary matter (SM) accounts for only $4\%$ of universe. What is \textcolor{red}{dark matter/dark energy}?
   \end{itemize}
    \centering
    \includegraphics[width=1.6in]{figs/intro/dark-matter.jpg}
      \vspace{0.4cm}
   \begin{itemize}
      \item {\em Compelling evidence for \textcolor{red}{beyond the SM} physics}. But how do we search? 
   \end{itemize}
}




\frame{
  \frametitle{Indirect searches}
   \vspace{-0.5cm}
   \begin{itemize}
     \item Historically, \textcolor{red}{indirect observations} of ``\textcolor{red}{new physics}'' has often been the portal to infer properties of heavy particles before experiments with sufficient energy to produce them. 
    \vspace{0.45cm}
     \item $\beta$ decay: particles of mass \textcolor{red}{$\sim 1$~GeV} reveals physics at \textcolor{red}{$\sim 100$~GeV}. 
   \end{itemize}
    \vspace{0.55cm}
    \centering
    \hspace{-0.5cm} \includegraphics[width=4.9in]{figs/intro/beta_decay_example.pdf}
}

\frame{
  \frametitle{BSM searches in (quark) flavor factories}
     \vspace{-0.4cm}
  \begin{columns}
    \column{0.55\textwidth}
   \begin{itemize}
     \item Core principle: \textcolor{red}{precision} tests of SM in processes that do not preserve quark flavor. 
     \vspace{0.3cm}
     \item Heavy \textcolor{red}{$\{b,c\}$} quarks decaying to lighter \textcolor{blue}{$q\in\{u,d,s\}$} quarks.  
   \end{itemize}
    \column{0.45\textwidth}
    \vspace{0.5cm}\\
    \centering
    \includegraphics[width=1.9in]{figs/intro/b2q_transition.pdf}\\
    $\mathcal{A} = \mathcal{A}_{\rm SM} + \mathcal{A}_{\rm BSM}$
  \end{columns}
     \vspace{0.2cm}
   \begin{itemize}
     \item \textcolor{red}{$\mathcal{A}_{\rm BSM} \propto \displaystyle  \frac{C_{BSM}}{M^2_{\rm BSM}}$} involves virtual heavy particles running inside \textcolor{red}{loops}.
     \vspace{0.4cm}
     \item Carefully choose scenarios where the \textcolor{red}{SM} part is \textcolor{red}{well-understood} and \textcolor{red}{suppressed} (helicity, Cabibbo, FCNC, ...)
     \vspace{0.4cm}
     \item Difference in rates or angular distributions wrt SM $\Rightarrow$ optimally chosen ``\textcolor{red}{clean}'' \textcolor{red}{observables} that reduce theory uncertainties. 
   \end{itemize}
}

{ % all template changes are local to this group.
    \setbeamertemplate{navigation symbols}{}
    \begin{frame}[plain]
        \begin{tikzpicture}[remember picture,overlay]
            \node[at=(current page.center)] {
                \includegraphics[width=\paperwidth]{figs/intro/epem_flavor_factories.pdf}
            };
        \end{tikzpicture}
     \end{frame}
}


{ % all template changes are local to this group.
    \setbeamertemplate{navigation symbols}{}
    \begin{frame}[plain]
        \begin{tikzpicture}[remember picture,overlay]
            \node[at=(current page.center)] {
                \includegraphics[width=\paperwidth]{figs/intro/lhc_flavor_factories.pdf}
                %\includegraphics[width=\paperwidth]{figs/intro/williams_lhc_aereal.jpg}
            };
        \end{tikzpicture}
     \end{frame}
}





















\frame{
  \frametitle{The LHCb detector}
   \vspace{-0.8cm}
  \begin{columns}
    \column{0.65\textwidth}
     \begin{itemize}
      \item Dedicated single-arm \textcolor{red}{forward} spectrometer with unique pseudo-rapidity range \textcolor{red}{$1.8 < \eta < 4.9$}.
     \vspace{0.1cm}
      \item $pp$ collisions in Run~1:
     \begin{itemize}
      \item 2011: 1/fb at 7~TeV
      \item 2012: 2/fb at 8~TeV
     \end{itemize}
     \vspace{0.1cm}
      \item High $b \bar{b}$ and $c \bar{c}$ cross-sections:
     \begin{itemize}
      \item $\sigma(pp \to b \bar{b})$ = $286~\mu$b at 7~TeV
      \item $\sigma(pp \to c \bar{c})$ = $\times 20$ larger
     \end{itemize}

     \end{itemize}
    \column{0.35\textwidth}
    \centering
    \includegraphics[width=1.9in]{figs/intro/08_rad_acc_scheme_right.pdf}
  \end{columns}

  \begin{columns}
    \column{0.4\textwidth}
     \begin{itemize}
      \item \textcolor{red}{Complementary} $\eta$ coverage wrt CMS/ATLAS. 
     \end{itemize}
    \column{0.6\textwidth}
    \centering
    \includegraphics[width=2.6in]{figs/intro/CMS_LHCb.jpg}
  \end{columns}
}


\frame{
  \frametitle{The LHCb subsystems and trigger}
  \begin{columns}
    \column{0.5\textwidth}
    \centering
    \includegraphics[width=2.8in]{figs/intro/lhcb_components.pdf}
    \column{0.4\textwidth}
    \includegraphics[width=2.8in]{figs/intro/LHCb_Trigger_Split.pdf}
  \end{columns}
     \begin{itemize}
      \item Tracking: VeLo + two tracking stations up- and down-stream of the magnet. Resolutions: 20~$\mu$m ($IP$), 0.5\% ($p$) and 45~fs ($\tau$). 
      \item Hadron Id: two RICH detectors with good $K$/$\pi$ seperation in $2 < p < 100$~GeV.
      \item Calorimetry: HCAL and ECAL for $\gamma, e, \piz$
      \item Muon detectors: 97\% efficiency, $<2.5\%$ $\pi\leftrightarrow \mu$ mis-ID 
     \end{itemize}
}


\section{$b\to s \ell \ell$}


\frame{
  \frametitle{The $b \to s \ellp \ellm$ ``industry'' at the LHC}
    \vspace{-0.8cm}

  \begin{columns}
    \column{0.65\textwidth}
   \begin{itemize}
    \item Everybody's favorite rare ``penguin'' decay!
    \vspace{0.4cm}
    \item Flavor-changing-neutral-current (FCNC).
    \vspace{0.4cm}
    \item No tree-level diagram in the \textcolor{blue}{SM}. Many ways where \textcolor{red}{NP} can enter. 
    \vspace{0.9cm}
    \item Several ways to explore this:
     \vspace{0.2cm}
    \begin{itemize}
      \normalsize
      \item \textcolor{red}{$B_s \to \mup \mun$} BF @ LHCb/CMS
      \vspace{0.2cm}
      \item \textcolor{red}{$B\to K^{\ast J}\; \gamma_{\rm pol}$} @ LHCb
      \vspace{0.2cm}
      \item \textcolor{red}{$B_d \to K^{(\ast)} \ellm \ellp$} @ LHCb 
      \vspace{0.2cm}
      \item $B_s \to \phi \mup \mun$, $\Lambda_b \to \Lambda \mup \mun$ ...
    \end{itemize}
   \end{itemize}
    \column{0.45\textwidth}
    \begin{center}
      \includegraphics[width=1.3in]{figs/intro/penguin_diag.pdf}\\
    \end{center}
    \hspace{-0.75cm} \includegraphics[width=2.2in]{figs/intro/b2smumu.pdf}\\
  \end{columns}

  %\begin{itemize}
  %  \vspace{0.3cm}
  %  \item \textcolor{red}{$B_s \to \mup \mun$}:
  %     \begin{itemize}
  %       \normalsize
  %         \vspace{0.2cm}
  %         \item Further \textcolor{red}{helicity suppressed}. BF $\sim \mathcal{O}(10^{-9})$.
  %         \vspace{0.2cm}
  %         \item highly \textcolor{red}{sensitive} ($(\tan \beta)^6$) to \textcolor{red}{extended Higgs} sector (esp. in SUSY context).  
  %     \end{itemize}
  %\end{itemize}
}


\frame{
  \frametitle{The Operator Product Expansion (OPE)}
  \begin{columns}
    \column{0.75\textwidth}
      \begin{itemize}
         \item Exactly as in the the case of Fermi's 4-point interaction theory of $\beta$-decay.
         \vspace{0.3cm}
         \item Expand $\mathcal{H}_{\rm eff}$ in a basis of local operators (OPE):
      \end{itemize}
    \column{0.3\textwidth}
     \includegraphics[width=1.4in]{figs/intro/b2smumu_4point.pdf}\\
  \end{columns}
     \vspace{0.5cm}
  \begin{columns}
    \column{0.75\textwidth}
     \centering
     \large
\begin{tcolorbox}[width=3.2in,height=1in,colback=yellow,colframe=red]
     \begin{align}
       \mathcal{H}_{\rm eff} &= - \displaystyle \frac{4 G_F}{\sqrt{2}} V_{tb} V^\ast_{ts} \displaystyle \sum_{i} \Big(\underbrace{C_i \mathcal{O}_i}_\text{had. LH} + \underbrace{C'_i \mathcal{O}'_i}_\text{had. RH} \Big) \;\;\;\;\nonumber
     \end{align}
\end{tcolorbox}
    \column{0.4\textwidth}
    \small
    \begin{tabular}{ c | l }
  i & Operator \\ \hline
  1,2 & Tree \\
  3-6,8 & Gluon Penguin \\
  $7_\gamma$ & Photon Penguin \\
  9,10 & EW Penguin \\
  S & Scalar \\
  P & Pseudoscalar 

\end{tabular} 
  \end{columns}

     \vspace{0.3cm}
      \begin{itemize}
         \item The \textcolor{red}{Wilson coefficients $C_i^{(')}(\alpha_s,\mu)$} encode short-distance physics, sensitive to $E \geq M_{EW}\sim M_W, M_Z$. Computed at $\mu \sim m_b$.
      \end{itemize}
}


\frame{
  \frametitle{Relevant operators in $b\to s \ellp \ellm$}
      \begin{itemize}
         \item $\mathcal{O}^{(')}(\mu)$ are composite operators depending on hadronic matrix element $\langle K^{(\ast)}\ellp \ellm,\ellp\ellm|\mathcal{H_{\rm eff}}|B\rangle$
         \vspace{0.5cm}
         \item \textcolor{red}{Tree-level} like with charm fields: $\mathcal{O}_1 \sim (\bar{s}_L \gamma_\mu c_L) (\bar{c}_L \gamma^\mu b_L)$
         \vspace{0.5cm}
         \item \textcolor{red}{Radiative} penguin: $\mathcal{O}_{7\gamma} \sim (\bar{s}_L \sigma_{\mu\nu} b)F^{\mu \nu}$
         \vspace{0.5cm}
         \item \textcolor{red}{Electroweak}: $\mathcal{O}_{9V} \sim (\bar{s}_L \gamma_\mu b_L) (\bar{\ell} \gamma^\mu \ell)$, $\mathcal{O}_{10A} \sim (\bar{s}_L \gamma_\mu b_L) (\bar{\ell} \gamma^\mu \gamma_5 \ell)$
         \vspace{0.5cm}
         \item \textcolor{red}{(pseudo)scalar}: $\mathcal{O}_S \sim (\bar{s}_L b_L) (\bar{\ell} \ell)$, $\mathcal{O}_P \sim (\bar{s}_L b_L) (\bar{\ell} \gamma_5 \ell)$. 
         \vspace{0.5cm} 
         \item Many more operators if one includes tensors (leptoquarks), etc...     
\end{itemize}
}


\frame{
  \frametitle{The Wilson coefficients}
      \vspace{-0.2cm}
  \begin{columns}
    \column{0.54\textwidth}
      \begin{itemize}
         \item Clues to \textcolor{red}{NP signature} hidden in the \textcolor{red}{$C_i$}'s.
      \end{itemize}
    \column{0.45\textwidth}
    \hspace{5cm}\\ \includegraphics[width=2.1in]{figs/intro/C_np_search.pdf}
  \end{columns}

       \vspace{0.2cm} 

      \begin{itemize}
         %\item \textcolor{red}{$C'_i$}: \textcolor{red}{RH hadronic} currents ($\propto \displaystyle \frac{m_s}{m_b} \sim 0.02$, \textcolor{red}{suppressed} in SM).
         %\vspace{0.5cm} 
         \item \textcolor{red}{SM hierarchy}: $C_{7\gamma} \sim -0.331$, $C_{9V} \sim 4.27$, $C_{10A}\sim -4.173$. Everything else small or negligible. 
         \vspace{0.4cm}
         \item Lots of complementarity in $C_{\rm NP}$ searches: 
         \vspace{0.2cm} 
         \begin{itemize}
           \normalsize
           \item $\mathcal{B}(\textcolor{blue}{B_s \to \ellp \ellm}) \sim m^2_\ell (C_{10A} - C'_{10A}) + (\textcolor{blue}{C_{S,P}} - C'_{S,P})$.
           \vspace{0.3cm} 
           \item \textcolor{red}{$\textcolor{DarkGreen}{B_d \to X_s \gamma_{\rm pol}}$}:  $\textcolor{DarkGreen}{C_{7\gamma}}$. Photon polarization: $C'_7$.
           \vspace{0.3cm} 
           \item \textcolor{red}{$B_d \to K^{(\ast)} \mup \mun$} angular analysis: \textcolor{red}{$C_{7\gamma}$, $C_{9V}$, $C_{10A}$} + ...
         \end{itemize}
      \end{itemize}
}


\frame{
  \frametitle{$B_{d,s} \to \mup \mun$}
    \vspace{-0.4cm} 
    \begin{itemize}
       \item After 30 years of search, $> 4~\sigma$ in LHCb and CMS for $B_s$.
           \vspace{0.3cm} 
       \item \textcolor{red}{LHCb and CMS combined} results ({\tt arXiv:1411.4413}, submitted to Nature): 
       \begin{itemize}
          \item $\mathcal{B}(\textcolor{blue}{B_d}) =3.9^{+1.6}_{-1.4}\times 10^{-10}$, \textcolor{blue}{$3.2~\sigma$} 
          \item $\mathcal{B}(\textcolor{red}{B_s}) = 2.8^{+0.7}_{-0.6}\times 10^{-9}$,\;\;  \textcolor{red}{$6.2~\sigma$ first observation} 
       \end{itemize}
    \end{itemize}
    \vspace{0.2cm} 
     \centering
     \includegraphics[width=5in]{figs/intro/Bds2mumu.pdf}
    \begin{itemize}
       \item Some slight tensions, but mostly \textcolor{red}{compatible with SM}.
    \end{itemize} 
}


\frame{
  \frametitle{$B_{d,s} \to \mup \mun$: effect on NP models}
    \begin{itemize}
       \item $\mathcal{B}(B_s\to \mup \mun)$ expected to be particularly enhanced ($\sim (\tan\beta)^6$, large $\tan \beta$) in the two-Higgs doublet models. 
         \vspace{0.1cm} 
       \item LHCb+CMS result has a \textcolor{red}{huge impact} on \textcolor{red}{SUSY} parameter space.  
    \end{itemize}
     \centering
     \includegraphics[width=3in]{figs/intro/susy_b2mumu.jpg}
}


\frame{
  \frametitle{Photon polarization in $b \to s \gamma$}
  \vspace{-0.3cm} 
  \begin{itemize} 
     \item Long history of radiative penguin $B \to X_s \gamma$ (inclusive) at CLEO, $\babar$, Belle, LHCb.
     \vspace{0.3cm} 
     \item Rate: $\mathcal{B}(b \to s \gamma) \propto |C_{7\gamma}|^2 + |C'_{7\gamma}|^2$, with $C'_{7\gamma}$ strongly suppressed in the SM.
     \vspace{0.2cm} 
     \item Novel feature: \textcolor{red}{outgoing photon} is almost fully \textcolor{red}{left-chiral} for \textcolor{red}{$b$ quark}. 
  \end{itemize} 
     \vspace{0.1cm} 
  \begin{columns}
    \column{0.4\textwidth}
     \centering
     \includegraphics[width=1.8in]{figs/intro/b2sgamma.pdf}
    \column{0.6\textwidth}
     \begin{itemize} 
       \item $\mathcal{A}_{\rm SM} \propto \textcolor{red}{m_b} \textcolor{red}{\bar{s}_L} \sigma_{\mu \nu} q^\nu \textcolor{red}{b_R} + \textcolor{blue}{m_s} \textcolor{blue}{\bar{s}_R} \sigma_{\mu \nu} q^\nu \textcolor{blue}{b_L}$ 
       \vspace{0.1cm} 
       \item $\displaystyle C'_{7\gamma}/C_{7\gamma} \sim \displaystyle \textcolor{blue}{m_s}/\textcolor{red}{m_b} \approx 0.02 \ll 1$
       \vspace{0.1cm} 
       \item $\lambda_\gamma = \displaystyle \frac{|C'_{7\gamma}|^2 - |C_{7\gamma}|}{|C'_{7\gamma}|^2 + |C_{7\gamma}|} = \textcolor{blue}{\underbrace{-1}_{\large b}}\;(\textcolor{red}{\underbrace{+1}_{{\overline{\normalsize b}}} })$  
      \end{itemize} 
  \end{columns}
     \vspace{0.15cm} 
     \begin{itemize} 
       \item \textcolor{red}{NP} can \textcolor{red}{enhance $C'_{7\gamma}$}: left-right symmetric models (w/ a heavy $W_R$).
      \end{itemize} 



}


\frame{
  \frametitle{Measurement of $\lambda_\gamma$ at LHCb}
  \vspace{-0.35cm} 
  \begin{columns}
    \column{0.9\textwidth}
     \begin{itemize} 
      \item Parity-odd triple product \textcolor{red}{$\vec{p}_\gamma \cdot(\vec{p}_\pi \times \vec{p}_\pi)$} in\\ \textcolor{red}{$B^\pm \to K^\pm \pi^\mp \pi^\pm \gamma$} decays is \textcolor{red}{sensitive to $\lambda_\gamma$}.
     \vspace{0.28cm} 
      \item Complicated Dalitz structures in $K\pi\pi$ system \\pushed into $C_{K\pi\pi} \sim 0.1$ (for $B^\pm$)
     \vspace{0.22cm} 
      \item \textcolor{red}{Up-down asymmetry} (arXiv:0205065): 
     \vspace{0.25cm} 
      \end{itemize} 
    \column{0.25\textwidth}
    \hspace{-2.3cm}\includegraphics[width=1.6in]{figs/intro/lambda_gamma.pdf}
  \end{columns}

  \begin{columns}
    \column{0.6\textwidth}
\vspace{-0.6cm}
     \begin{align}
      \textcolor{red}{A_{UD}} \equiv \frac{N_{\cos\theta > 0} - N_{\cos\theta < 0}  }{N_{\cos\theta > 0} + N_{\cos\theta < 0}} = C_{K\pi\pi} \textcolor{red}{\lambda_\gamma} \nonumber
     \end{align}
\pause
\centering
\begin{tcolorbox}[width=2.2in,height=0.47in,colback=yellow,colframe=red]
\vspace{-0.1cm}
 First observation of photon\\ polarization in $b\to s \gamma$ 
\end{tcolorbox}
\vspace{-0.1cm}
     \begin{itemize} 
      \item Further theory input needed on $C_{K\pi\pi}$ to extract $\lambda_\gamma$. 
      \end{itemize} 
    \column{0.6\textwidth}
     \vspace{0.1cm} 
    \includegraphics[width=2.1in]{figs/intro/lambda_gamma_result.pdf}
  \end{columns}
}


\frame{
  \frametitle{Angular analyses of $\Bbar \to X \ell_1 \ell_2$}
     \vspace{-0.4cm} 
     \begin{itemize}
        \item So far so good. No spectacular deviations from SM. 
        \vspace{0.2cm} 
        \item As we will see, the ``interesting'' anomalies involve $\Bbar \to X \ell_1 \ell_2$.
         \begin{itemize} 
            \normalsize
            \item Electroweak Penguins (\textcolor{red}{EWP}): \textcolor{red}{$\ellm \ellp$}, $\ell \in \{e,\mu\}$
        \vspace{0.2cm} 
            \item Semileptonic (\textcolor{blue}{SL}): \textcolor{blue}{$\ellm \barnuell$}, $\ell \in \{e,\mu, \tau\}$
        %\vspace{0.2cm} 
        %    \item Charmonia (\textcolor{blue}{$C\overline{C}$}): \textcolor{blue}{$J/\psi\to \ellp \ellm$}, $\ell \in \{e,\mu\}$
        \vspace{0.2cm} 
            \item $X$ is a mesonic system $\in \{\pi, K, \;\pi \pi,\; K\pi, \;KK, \;D, \;D\pi\}$
         \end{itemize}
     \end{itemize} 
     \begin{itemize}
         \item Four kinematic variables: $\phi \in \{\textcolor{red}{\qsq, \thetal, \thetav, \chi}\}$
     \end{itemize}
     \centering 
    \includegraphics[width=2.5in]{/home/hep/biplabd/wdir/babar/notes/pubs/prd_angular/ver3/figs/btorhoellnu_angle_vars.pdf}
}


%\frame{
%   \frametitle{Common suite of angular analysis tools}
%   \begin{itemize}
%      \item Rich angular structure means large number of observables to probe NP.
%      \item Common suite of analysis tools to handle EWP, SL and $B_{d,s} \to J/\psi (K^\ast,\phi)$ (for $\beta_{(s)}$).  
%   \end{itemize}

%\begin{align}
%\overline{C}^{L,R} &\equiv \left[ (C^{\rm eff}_9 - {C^{\rm eff}}'_9) \mp (C^{\rm eff}_{10} - {C^{\rm eff}}'_{10}) \right] /2 \nonumber \\
%\overline{C}'^{L,R} &\equiv \left[ (C^{\rm eff}_9 + {C^{\rm eff}}'_9) \mp (C_{10} + {C^{\rm eff}}'_{10}) \right] /2 \nonumber \\
%\overline{C}_{7} &\equiv (C^{\rm eff}_7 - {C^{\rm eff}}'_7)/2 \nonumber \\
%\overline{C}'_{7} &\equiv (C^{\rm eff}_7 + {C^{\rm eff}}'_7)/2 \nonumber
%\end{align}
%   \begin{itemize}
%      \item 
%   \end{itemize}
%}




\frame{
  \frametitle{The helicity amplitudes}
     \vspace{-0.5cm}
     \begin{itemize}
         \item Three helicity amplitudes for the \textcolor{red}{spin-1 dilepton} $\{W^\ast, Z^\ast, \gamma^{(\ast)}\}$: 
     \end{itemize}
\scriptsize
\begin{align}
\!\!H^{L,R} \Big|_{J=0} &=   \frac{2 m_B {\bf k}  }{\sqrt{\qsq}}\left\{  \textcolor{red}{\overline{C}^{L,R}} \textcolor{blue}{F_1(\qsq)} + \textcolor{red}{\overline{C}_{7}} \frac{2 m_B}{m_B+m_X} \textcolor{blue}{F_T(\qsq)} \right\} \nonumber \\
H^{L,R}_\pm \Big|_{J\geq1} &= \beta_J  \left( \frac{{\bf k}}{m_X} \right)^{J-1} \Bigg\{ \Bigg. \textcolor{red}{\overline{C}^{L,R}} (m_B+m_X) \textcolor{blue}{A_1(\qsq)} + \frac{2 m_B}{\qsq} (m^2_B-m^2_X) \textcolor{red}{\overline{C}_{7}} \textcolor{blue}{T_2(\qsq)}  \Bigg. \nonumber\\
& \hspace{3cm} \Bigg. \mp 2 m_B {\bf k} \left[  \textcolor{red}{\overline{C}'^{L,R}} \frac{\textcolor{blue}{V(\qsq)}}{m_B+m_X} + \textcolor{red}{\overline{C}'_{7}} \frac{2 m_B}{\qsq} \textcolor{blue}{T_1(\qsq)} \right] \Bigg\} \nonumber \\
H^{L,R}_0\Big|_{J\geq1}\! &= \!\frac{\alpha_J}{2 m_X \sqrt{\qsq}}  \left( \frac{{\bf k}}{m_X} \right)^{J-1} \Bigg\{ \textcolor{red}{\overline{C}^{L,R}} \left[ (m^2_B - m^2_X - \qsq)(m_B+m_X) \textcolor{blue}{A_1(\qsq)} - \frac{4 m^2_B {\bf k}^2}{m_B+m_X}\textcolor{blue}{A_2(\qsq)} \right] \nonumber \\
&  \hspace{3cm} \Bigg.+ 2 m_B \textcolor{red}{\overline{C}_7} \left[ (m^2_B + 3 m^2_X - \qsq) \textcolor{blue}{T_2(\qsq)} - \frac{4 m^2_B {\bf k}}{m^2_B - m^2_X} \textcolor{blue}{T_3(\qsq)} \right] \Bigg\} \nonumber
\end{align}

\normalsize
     \begin{itemize}
         \item \textcolor{blue}{QCD form-factors} are the largest source of systematic uncertainties. 
         %\vspace{0.1cm}
         %\item \textcolor{green}{SL} case: \textcolor{red}{$\overline{C}^R = \overline{C}'^R = \overline{C}_7 = \overline{C}'_7=0$}, \textcolor{red}{$\overline{C}^L = \overline{C}'^L = 1$}. 
     \end{itemize}
}

\frame{
   \frametitle{The ``clean'' observables}
     \begin{itemize}
         \item If the $X$ system is in spin-$J$, the amplitude squared reads: 
     \end{itemize}
\begin{align}
|\overline{\mathcal{M}}|^2 &= \sum_{L,R}  \Bigg| \sum_{\lambda\in \{0,\pm 1\}} \sum_J \sqrt{2J+1} \mathcal{H}^{\{L,R\},J}_\lambda d^J_{\lambda,0} (\thetaV) d^1_{\lambda,\eta} (\thetal) e^{i\lambda \chi} \Bigg|^2 \nonumber \\
                           &= \displaystyle \sum_i \textcolor{red}{\Gamma_i(\qsq)} \; f_i(\thetal,\thetaV,\chi) \nonumber
\end{align}

     \begin{itemize}
         \item Matias {\em et al.} (arXiv:1303,5794): carefully constructed \textcolor{red}{ratios} of the \textcolor{red}{$\Gamma_i$ observables}.
     \vspace{0.2cm}
         \item Leading order FF uncertainties cancel in the \textcolor{red}{$\qsq \leq 6$~GeV$^2$} regime.
     \vspace{0.1cm}
         \item \textcolor{red}{Forward-backward zero crossing} point long known to be theoretically clean.
     \vspace{0.1cm}
         \item \textcolor{red}{New observable $P'_5$} turns out to be particularly sensitive. 
     \end{itemize}
}



\frame{
  \frametitle{$B^0 \to K^{\ast 0}\mup\mun$ status with 1/fb data}
     \vspace{-0.3cm}
     \begin{itemize}
       \item Good agreement with the SM (JHEP 07 (2011) 067) 
     \end{itemize}
    \centering
    \includegraphics[width=5in]{figs/intro/kstmumu_1ifb.pdf}
     \pause
     \begin{itemize}
       \item Except \textcolor{red}{$3.7\sigma$ local deviation} in \textcolor{red}{$P'_5$} from SM (JHEP 05 (2013) 137): 
     \end{itemize}
    \centering
    \includegraphics[width=2in]{figs/intro/effing_nico.pdf}
}


\frame{
  \frametitle{3/fb: anomalous trends now seen in $\mathcal{B}$}
     \vspace{-0.3cm}
     \begin{itemize}
       \item \textcolor{red}{Isospin} channels: \textcolor{red}{$\mathcal{B}$} tend to lie \textcolor{red}{below SM} (PRL 112 212003, arXiv:1411.3161) 
     \end{itemize}
     \vspace{0.1cm}
    \centering
    \includegraphics[width=4.9in]{figs/intro/b2kstmumu_isospin.pdf}
     \vspace{0.35cm}
    \pause
  \begin{columns}
    \column{0.4\textwidth}
     \begin{itemize}
       \item Updated lattice calculations at \textcolor{red}{high $\qsq$} reflect this as well 
     \end{itemize}
    \column{0.7\textwidth}
    \includegraphics[width=3.2in]{figs/intro/wingate.pdf}
  \end{columns}
}


\frame{
  \frametitle{Effect on Wilson coefficients}
  \begin{columns}
    \column{0.4\textwidth}
    \includegraphics[width=2in]{figs/intro/c9.pdf}
    \column{0.6\textwidth}
     \begin{itemize}
       \item With 2013 data, \textcolor{red}{$C_9$} and \textcolor{red}{$C_7$} seen as main players. 
        \vspace{0.3cm}
       \item All theory groups find \textcolor{red}{$C_9^{\rm NP} < 0$}\\ \vspace{0.05cm}
\scriptsize
Altmannshofer, Straub 1308.1501,
Beaujean, Bobeth, van Dyk 1310.2478,
Horgan et al. 1310.3887
Hambrock, Hiller, Schacht, Zwicky 1308.4379.
\normalsize
       \item Different bins, observables, statistical approaches...
        \vspace{0.3cm}
       \item Contentions whether $C'_9 \sim - C_9^{\rm NP}$ or $\textcolor{blue}{C'_9 \to 0}$ (\textcolor{blue}{Matias} et al.)
        \vspace{0.3cm}
     \end{itemize}
  \end{columns}
}


\frame{
    \frametitle{Another unexpected development: $R_K$} 
     \begin{itemize}
      \item Other than a tiny effect from mass difference, $e/\mu$ behaves the same in SM
     \end{itemize}
  \begin{columns}
    \column{0.7\textwidth}
     \begin{itemize}
      \item $R_K \equiv \displaystyle \frac{\mathcal{B}(B \to K\mu \mu)}{\mathcal{B}(B \to Ke e)} = 1.0 \pm \mathcal{O}(10^{-4})$
       \vspace{0.7cm}
      \item \textcolor{red}{$R^{\rm LHCb}_K =0.745^{+0.090}_{-0.074} \pm 0.036$}
     \end{itemize}
    \column{0.3\textwidth}
    \hspace{-1.75cm}\includegraphics[width=2in]{figs/intro/rk.pdf}
  \end{columns}

\vspace{0.5cm}
\centering
\begin{tcolorbox}[width=3.3in,height=0.47in,colback=yellow,colframe=red]
\vspace{-0.08cm}
Hints towards lepton universality violation in \\1st and 2nd generations for the first time
\end{tcolorbox}
}

\frame{
    \frametitle{Angular analysis of $B^0 \to K^\ast e^+ e^-$ at low $\qsq$}
   
     \begin{itemize}
      \item Since $q^2_{\rm min} \geq 4m^2_\ell$, $ee$ mode allows to explore the \textcolor{red}{very low $\qsq$} region.
     \end{itemize}
       \vspace{0.3cm}

  \begin{columns}
    \column{0.7\textwidth}
     \begin{itemize}
      \item Sensitivity to \textcolor{red}{$C^{(')}_{7\gamma}$} competitive with\\ rad. penguins. 
       \vspace{0.3cm}
      \item New LHCb results: angular analysis \\in $\qsq\in [0.002,1.120]$~GeV$^2$.
       \vspace{0.3cm}
       \item Results \textcolor{red}{consistent} with \textcolor{red}{SM}. 
     \end{itemize}
    \column{0.3\textwidth}
    \hspace{-1.75cm}\includegraphics[width=2in]{figs/intro/kstee.pdf}
  \end{columns}

\vspace{0.5cm}
     \begin{itemize}
        \item \textcolor{red}{$R_{K^\ast}$} could be very interesting
       \vspace{0.3cm}
        \item Experimentally much more challenging than $\mu\mu$: trigger and modeling of \textcolor{red}{bremsstrahlung} inside detector material.
     \end{itemize}
}


\frame{
\centering
\begin{tcolorbox}[width=3.7in,height=1in,colback=white,colframe=black]
\vspace{-0.08cm}
{\Huge \textcolor{DarkGreen}{$B^0\to K^\ast \mup \mun$ update\\ from Moriond 2015}}
\end{tcolorbox}
}

\frame{
  \frametitle{The anomaly persists!}

    \centering
    \includegraphics[width=3in]{figs/intro/b2ksmumu_moriond.pdf}
     \begin{itemize}
      \item Excellent \textcolor{red}{consistency} between \textcolor{red}{2011} and \textcolor{red}{2012} results.
     \end{itemize}
}

\frame{
  \frametitle{And if we are aggressive...}

    \centering
    \includegraphics[width=2.7in]{figs/intro/b2ksmumu_moriond_c9np.pdf}
       \vspace{0.6cm}
     \begin{itemize}
      \item Moves \textcolor{red}{closer} to \textcolor{red}{$C^{\rm NP}_9 < 0$} as well. 
     \end{itemize}
}

\frame{
  \frametitle{Zero-crossing point in $A_{FB}$}
    \centering
    \includegraphics[width=2.7in]{figs/intro/b2ksmumu_moriond_zcp.pdf}
       \vspace{0.6cm}
     \begin{itemize}
      \item Slightly lower than SM. ZCP: $q^2_0 \sim 3.7$~GeV$^2$.
     \end{itemize}
}


\frame{
  \frametitle{Heavy $Z'$ with non-universal flavor couplings?}
   \vspace{-0.5cm}
   \begin{itemize}
      \item Numerous theory papers combining all $b \to s \ell \ell$ measurements\\ {\tiny Descotes-Genon et al [1307.5683], Beaujean et al [1310.2478], Gauld et al [1308.1959], Hurth et al [1312.5267], Straub et al [1308.1501], Horgan et al [1310.3887],Altmannshofer et al [1403.1269], Biancofore et al [1403.2944]...}
       \vspace{0.3cm}
      \item Simplest explanantion of \textcolor{red}{$C^{\rm NP}_9 \sim -1.5$} is a $Z'$ boson with specific flavor couplings.
   \end{itemize}

  \begin{columns}
    \column{0.8\textwidth}
     \begin{itemize}
       \item \textcolor{red}{Heavy} (TeV range) \textcolor{red}{$Z'$ boson} with \textcolor{red}{FCNC at tree-level} 
       \vspace{0.3cm}
       \item Couples only to \textcolor{red}{LH quarks}.
       \vspace{0.3cm}
       \item Couples equally with $\ell_{R,H}$, but \textcolor{red}{differently to $e/\mu$}
       \vspace{0.3cm}
       \item CPV in $B_s$-$\overline{B}_s$ places strong limits on the couplings.
       \vspace{0.3cm}
       \item Difficult to accomodate within MSSM.
     \end{itemize}
    \column{0.3\textwidth}
    \hspace{-0.5cm} \includegraphics[width=1.4in]{figs/intro/zprime.pdf}
   \end{columns}
}


\frame{
  \frametitle{Charm-loop effects: potential show-stopper?}
     \begin{itemize}
       \item Lyon-Zwicky (1406.0566): \textcolor{red}{non-factorizable $c\bar{c}$} loops are \textcolor{red}{large} and can accomodate the $P'_5$ anomaly. 
     \end{itemize}

   \vspace{-0.5cm}
  \begin{columns}
    \column{0.7\textwidth}
     \begin{itemize}
       \item OPE breaks down. Very hard to calculate. 
       \vspace{0.7cm}
       \item Can we disentangle $b \to s \ell \ell$ from $b\to s c\bar{c}$? 
     \end{itemize}
    \column{0.3\textwidth}
     \includegraphics[width=1.5in]{figs/intro/ccbar_nonFT.pdf}
   \end{columns}

     \begin{itemize}
       \item What can we do experimentally in the $b \to s \mup \mun$ sector?  
       \begin{itemize}
          \normalsize
          \item Go \textcolor{red}{closer} to \textcolor{red}{$\qsq = m^2_\psi$} by improving $\psi$ vetoes.
       \vspace{0.3cm}
          \item Look at \textcolor{red}{higher $K^{\ast J}$ states}, especially around the $K^\ast_2(1430)$.
       \end{itemize}
     \end{itemize}
}

\section{$\Vub$ and $\Vcb$}


\frame{
  \includegraphics[width=2in]{figs/intro/Somethingdifferent.jpg}\hspace{1cm}
  \includegraphics[width=2in]{figs/intro/tree_b2q.pdf}
}


\frame{
  \frametitle{Importance of $\Vub/\Vcb$}

   \begin{columns}
    \column{0.4\textwidth}
     \;\;$ \textcolor{red}{\vckm} \equiv  \begin{pmatrix} \vud & \vus & \vub \\ \vcd & \vcs & \vcb \\ \vtd & \vts & \vtb \end{pmatrix}$
    \column{0.9\textwidth}
    \begin{itemize}
      \item In SM, \textcolor{red}{unitary CKM} matrix $\Rightarrow$ \textcolor{red}{flavor-mixing}
      \vspace{0.2cm}
      \item Fantastic success of the CKM paradigm \\thru' the years... 
    \end{itemize}
   \end{columns}

    \vspace{0.4cm}

  \begin{columns}
    \column{0.5\textwidth}
    \centering
    \underline{\textcolor{red}{1995}:}\\
    \vspace{0.05cm}
    \includegraphics[width=1.8in]{figs/intro/rhoeta_large_global_1995.eps}
    \column{0.5\textwidth}
    \centering
    \underline{\textcolor{red}{2014}:}\\
    \vspace{0.05cm}
    \includegraphics[width=1.8in]{figs/intro/rhoeta_large.eps}
  \end{columns}
}



\frame{
  \frametitle{Importance of $\Vub/\Vcb$}

    \begin{itemize}
      \item Side opposite to $\beta$ proportional to $\Vub/\Vcb$. Both $\beta$ and $\Vcb$ known better than $3\%$. 
      \vspace{0.4cm}
      \item \textcolor{red}{Closure test of UT} mainly limited by \textcolor{red}{$\Vub$}.
    \end{itemize}

      \vspace{0.4cm}

    \centering
    \includegraphics[width=4in]{figs/intro/rhoeta_small_global.eps}
}

\frame{
   \frametitle{Inclusive and exclusive $\Vub$}
    \vspace{-0.4cm}
    \begin{itemize}
      \item \textcolor{red}{Exclusive $\Bbar \to \pi \ellm \barnuell$}. Need QCD form-factors.
      \vspace{0.2cm}
      \item \textcolor{DarkGreen}{Inclusive} $\Bbar \to X_u \ellm \barnuell$. No form-factors (\textcolor{DarkGreen}{sum of states}), but kinematics cuts to reduce $\times 50 $ large charm background.
      \vspace{0.2cm}
      \item Different experiment/theory techniques. \textcolor{magenta}{Persistent $\sim 3\sigma$ tension}! 
    \end{itemize}
      \vspace{0.2cm}
    \centering
    \includegraphics[width=2.2in]{figs/intro/incl_excl.pdf}
}

\frame{
   \frametitle{$\Vub$ at LHCb via $\Lambda_b \to p \mu \nu$}

   \centering
\begin{tcolorbox}[width=4.4in,height=0.35in,colback=yellow,colframe=red]
      \vspace{-0.05cm}
      $\Vub/\Vcb$ long thought to be impossible at a hadron collider
\end{tcolorbox}

    \begin{itemize}
      \item LHCb probes $b \to u $ in \textcolor{red}{exclusive} baryonic \textcolor{red}{$\Lambda_b \to p \mu \nu$} decay. 
      \vspace{0.2cm}
      \item High statistics ($\mathcal{O}(10^4)$) even for a rare decay!
      \vspace{0.2cm}
      \item Critial role played by \textcolor{red}{latest lattice} calculations at \textcolor{red}{high $\qsq$}.  
    \end{itemize}
    \centering
      \vspace{0.7cm}
    \includegraphics[width=4.7in]{figs/intro/wingate_1.pdf}
}

\frame{
   \frametitle{Two experimental challenges}
      \vspace{-0.8cm}
  \begin{columns}
    \column{0.7\textwidth}
    \begin{itemize}
      \item Dominant \textcolor{blue}{charm backgrounds} have additional tracks close to the $p\mu$ vertex.
      \item Multivariate classifier trained to discriminate between \textcolor{red}{red} and \textcolor{blue}{blue} tracks.
      \item $90\%$ rejection with $10\%$ efficiency.
    \end{itemize}
    \column{0.3\textwidth}
      \vspace{0.4cm}
    \includegraphics[width=1.3in]{figs/intro/isolation.pdf}
  \end{columns}


  \begin{columns}
    \column{0.45\textwidth}
    \begin{itemize}
      \item \textcolor{red}{Two-fold ambiguity in $\qsq$}: 
      %\item Fit the corrected mass:\\$M_{\rm corr} = \sqrt{p^2_\perp + M^2_{p\mu} } + p_\perp$. 
    \end{itemize}
     \vspace{0.2cm}
    \centering
    \includegraphics[width=2.2in]{figs/intro/ambiguity.pdf}
    \column{0.55\textwidth}

    \begin{itemize}
      \item Require both $\qsq$ solns. $\geq 15$~GeV$^2$:  
    \end{itemize}
    \centering
    \includegraphics[width=1.8in]{figs/intro/q2_eff.pdf}
  \end{columns}
}

\frame{
   \frametitle{$M_{\rm corr}$ fits}
    \begin{itemize}
      \item Fit the \textcolor{red}{corrected mass}: $M_{\rm corr} = \sqrt{p^2_\perp + M^2_{p\mu} } + p_\perp$. 
    \end{itemize}
    \vspace{0.8cm}
    \centering
    \includegraphics[width=4in]{figs/intro/mcorr_fits.pdf}
}


\frame{
   \frametitle{What can LHCb say about $\Vub$?}

    \begin{itemize}
      \item $\Vub_{\rm LHCb} = (3.27 \pm 0.15 ({\rm exp}) \pm  0.17({\rm theory}) \pm  0.06(\Vcb))\times 10^{-3}$ \\ ({\tt LHCB-PAPER-2015-013})
    \end{itemize}
      \vspace{0.2cm}
  \begin{columns}
    \column{0.6\textwidth}
    \begin{itemize}
      \item Total uncertainty is $7.2\%$. \textcolor{red}{World's best exclusive measurement}.
      \vspace{0.4cm}
      \item Consistent with WA of $\sin (2\beta)$. 
    \end{itemize}
    \column{0.4\textwidth}
    \includegraphics[width=1.5in]{figs/intro/vub_lhcb.pdf}
  \end{columns}

      \vspace{0.5cm}
   \centering
\begin{tcolorbox}[width=4.3in,height=0.35in,colback=yellow,colframe=red]
      \vspace{-0.05cm}
      Large tension between inclusive and exclusive \Vub persists
\end{tcolorbox}
}

\frame{
   \frametitle{Right-handed currents in the SL sector}
     \vspace{-0.5cm}
     \begin{itemize}
       \item In the \textcolor{blue}{SM}, weak interaction is \textcolor{blue}{purely LH}. Add a small \textcolor{red}{RH admixture $\epsR$} (connected to $C'$ variables). 
             \vspace{0.3cm}
       \item If \textcolor{red}{$\epsR$ real and negative}, can help resolve the $\Vub$ tension:
          \normalsize
          \begin{itemize}
             \item $\Vub_{incl.} \sim 1 +\epsR^2$   
             \item $\Vub_{excl.} \sim 1 +\epsR$   
          \end{itemize}
     \vspace{0.2cm}
        \item Several authors have looked at this (1408.2516, 1411.1177, 1407.1320 etc.) but no clear picture.
     \end{itemize}


  \begin{columns}
    \column{0.75\textwidth}
          \begin{itemize}
       \item $\epsR$ difficult to decouple from the FF normalizations.
             \vspace{0.4cm}
       \item Similar \textcolor{red}{$3\sigma$ tensions} seen in the \textcolor{red}{$\Vcb$ sector} as well.
          \end{itemize}
    \column{0.25\textwidth}
    \centering
    \includegraphics[width=1in]{figs/intro/gambini.pdf}
  \end{columns}
}



\frame{
   \frametitle{The BReco technique at $B$-factories}
   \vspace{-0.3cm}
   \begin{columns}
      \column{0.55\textwidth}
         \vspace{-0.1cm}
         \begin{itemize}
            \item In $e^+ e^- \to \FourS \to \bsig\btag$,\\ \textcolor{red}{full hadronic reconstruction} of $\btag$
            \vspace{0.6cm}
            \item A \textcolor{red}{single missing $\nu$}: reconstructed\\ as $p_{\scriptsize \text{miss}}$ via kinematic fit.
         \end{itemize}
      \column{0.5\textwidth}
         \includegraphics[width=2.3in]{/home/hep/biplabd/wdir/babar/talks/lhcb_zurich/figs/intro/breco.pdf}
   \end{columns}
            \vspace{0.7cm}

         \begin{itemize}
            \item $\Bbar \to X \mun \mup$ (LHCb) and $\Bbar \to X \ellm \barnuell$ ($\babar$) on a completely equal footing now (resolutions).  
            \vspace{0.5cm}
            \item Low efficiency, but de facto method in Belle~II era for neutrinos.
         \end{itemize}

}

\section{Taonic $B$ decays}


\frame{
   \frametitle{$\Bbar \to D^\ast \ellm \barnuell$ at $\babar$ }
  \begin{columns}
    \column{0.5\textwidth}
    \centering
    \includegraphics[width=2.2in]{/home/hep/biplabd/wdir/babar/talks/lhcb_zurich/figs/data_anal/bdslnu.pdf}
    \column{0.5\textwidth}
    \begin{itemize}
      \item Remarkably clean dataset!
       \vspace{0.2cm}
    \end{itemize}
     \centering
      \fcolorbox{red}{yellow}{$\sim 6000$ events w/ $\sim 2\%$ bkgd.}
    \begin{itemize}
       \vspace{0.2cm}
      \item Pure $P$-wave. Best known FF's from HQET.
    \end{itemize}
   \end{columns}

    \begin{itemize}
      \item Entire suite of ``clean'' observables accessible here.
       \vspace{0.3cm}
      \item Any \textcolor{red}{$\sin\chi$} term (absent in SM) sensitive to \textcolor{red}{$Im(\epsR)$}. 
       \vspace{0.3cm}
       \item Cleanest/simplest possible situation $\Rightarrow$ very important testbed.
    \end{itemize}
}






\frame{
  \frametitle{The $\tau$ case is special}
     \vspace{-0.5cm}
     \begin{itemize}
       \item For \textcolor{red}{massive $\tau$}, chirality $\neq$ helicity. The $W^\ast$ can have \textcolor{red}{spin-0}. 
     \end{itemize}
  \begin{columns}
    \column{0.6\textwidth}
     \begin{itemize}
       \item \textcolor{red}{Charged-Higgs} enters at \textcolor{red}{tree-level}.
       \vspace{0.3cm}
       \item Type 2 two-Higgs doublet model: \textcolor{red}{amplitude} scales as \textcolor{red}{$m_\tau\left(\frac{\tan \beta}{m_{H^\pm}}\right)^2$}.
       \vspace{0.2cm}
     \end{itemize}
    \column{0.5\textwidth}
    \includegraphics[width=1.8in]{figs/intro/b2dtaunu.pdf}
  \end{columns}
     \begin{itemize}
       \item Additional form-factor: $A_0(\qsq)$. Different phase-space constraints:
     \end{itemize}
     \vspace{0.2cm}
    \centering
    \includegraphics[width=3.5in]{figs/intro/tau_manuel.pdf}
}

\frame{
  \frametitle{The $\babar$ $R(D^{(\ast)})$ anomaly (1205.5442)}
  \begin{columns}
    \column{0.73\textwidth}
   \begin{itemize}
     \vspace{-0.3cm}
     \item Measured the ratios:\\ $R(D^{(\ast)}) \equiv \displaystyle \frac{\mathcal{B}(B \to D^{(\ast)}\tau \nu   )}{\mathcal{B}(  B \to D^{(\ast)}\ell \nu  ) }$, $\tau \to \ell \barnuell \nu_\tau$. 
     \vspace{0.5cm}
     \item Same final state in numerator and denominator. Many uncertainties cancel.
     \vspace{0.5cm}
     \item $R(D^{(\ast)}) > 3~\sigma$ tension with SM:
   \end{itemize}
    \centering
    \includegraphics[width=2.65in]{figs/intro/manuel_3.pdf}
    \column{0.3\textwidth}
    \centering
     2-d fit in $p^2_{\rm miss}$ and $p^\ast_\ell$:\\ 
     \vspace{0.6cm}
    \includegraphics[width=1.2in]{figs/intro/manuel1.png}
  \end{columns}
}

\frame{
  \frametitle{2HDM models confronting $\babar$ data}
  \begin{columns}
    \column{0.5\textwidth}
    \centering
    \includegraphics[width=2.2in]{figs/intro/manuel_2hdm.png}
    \column{0.5\textwidth}
   \begin{itemize}
      \item \textcolor{red}{type II 2HDM}: the two Higgs doublets couple to up- and down-type quarks separately.
      \vspace{0.5cm}
      \item Favored scenario in MSSM.
      \vspace{0.5cm}
      \item However, can't explain $R(D)$ and $R(D^\ast)$ \textcolor{red}{simultaneously}. 
   \end{itemize}
  \end{columns}

   \vspace{0.5cm}
   \begin{itemize}
      \item Can be accomodated in \textcolor{red}{type III 2HDM} (both doublets couple to up and down-type quarks). Crivellin et al. (1206.2634v2).
   \end{itemize}
}

\frame{
  \frametitle{Connection of $R(D^{(\ast)})$ models to $\epsR$}
    \begin{itemize}
      \item Effective Lagrangian approach with genaric 4-quark operators (Datta et al., 1206.3760) includes $g_V$, $g_A$, $g_S$, $g_P$...
      \vspace{0.2cm}
      \item But $\displaystyle \frac{g_V}{g_A} \sim \frac{1+ \epsR}{1-\epsR}$, so this should affect $\ell \in \{e, \mu\}$ cases as well. 
    \end{itemize}
    \centering
    \includegraphics[width=3in]{figs/intro/gv_ga.pdf}
    \begin{itemize}
      \item Angular analysis of $\Bbar \to D^\ast \ellm \barnuell$ should place constraints on these models. 
    \end{itemize}
     
}


\frame{
  \frametitle{$R(D^{(\ast)})$ from Belle and LHCb}
  \begin{columns}
    \column{0.4\textwidth}
    \centering
    \includegraphics[width=2in]{figs/intro/BOZEK.pdf}
    \column{0.6\textwidth}
    \begin{itemize}
      \item We (still) eagerly await results from the full Belle~I dataset.
      \vspace{0.7cm}
      \item ``\textcolor{red}{Trends}'' indicate \textcolor{red}{$\babar$ + Belle~I} might be \textcolor{red}{close to $5~\sigma$}! 
      \vspace{0.8cm}
      \item LHCb has also entered the game.
    \end{itemize}
  \end{columns}
}

\section{Summary}

\frame{
   \frametitle{Summary and outlook}
    \vspace{-0.3cm}
    \begin{itemize}
      \item We are in a unique situation where \textcolor{red}{several very interesting tensions} exist with the SM in the \textcolor{red}{heavy quark flavor sector}.
      \vspace{0.1cm}
      \item Happening times -- both Run~II at LHC and Belle~II. {\em Much} \textcolor{red}{more data} expected \textcolor{red}{soon}. 
      \vspace{0.1cm}
\pause
      \item The higher $\Lambda_{\rm NP}$ is, the more unexpected non-CKM type flavor violations. \textcolor{red}{Measure everything}!
     \end{itemize}
     \centering
    \includegraphics[width=2.2in]{figs/intro/nima.pdf}
}

\section*{Back-up slides}


\addtocounter{framenumber}{-1}

\frame{
   \frametitle{LHCb upgrade}
    \centering
    \includegraphics[width=3.8in]{figs/intro/upgrade.pdf}
     \begin{itemize}
       \item Higher track multiplicities, ghost rates, interactions/crossing, vertices...
       \item Replace current hardware L0 trigger (1MHz) to more flexible software tigger
       \item Read out everything (40~MHz) and HLT output 20~kHz
       \item VeLo and tracking (new Upstream Tracker, Fiber Tracker)
       \item RICH system: new photo-detectors, upgraded optics
     \end{itemize}
}

\addtocounter{framenumber}{-1}

\frame{
   \frametitle{LHCb upgrade}
    \centering
    \includegraphics[width=4.9in]{figs/intro/upgrade_obs.pdf}

}

\end{document}
    
