


\frame{

   \frametitle{$B\to J/\psi K\pi$: $m(K\pi)$ spectrum}

   \begin{columns}
      \column{0.5\textwidth}
      \centering
      {\large \textcolor{red}{Data}}:\\
      \includegraphics[width=2.3in]{figs/intro/jpsikpi_data.pdf}
      \column{0.5\textwidth}
      \centering
      {\large \textcolor{blue}{PHSP MC}}:\\
      \includegraphics[width=2.3in]{figs/intro/jpsikpi_phsp.pdf}
   \end{columns}
\vspace{0.3cm}
  \begin{itemize}
    \item This is what the \textcolor{red}{Data} looks like, as is. \textcolor{red}{Irrespective} of the production \textcolor{red}{dynamics} or analysis method.
\vspace{0.3cm}
    \item \textcolor{red}{Data}: dominated by the $K^\ast(892)$ and the $K^{(\ast,\ast \ast)}(1430)$'s. Very \textcolor{red}{little} statistics \textcolor{red}{beyond $\sim$~1500~MeV}.
  \end{itemize}
}

\frame{
   \frametitle{Using {\tt PHSP} generator}
  \begin{itemize}
     \item {\tt PHSP} generator has two problems:
  \begin{itemize}
      \normalsize
\vspace{0.1cm}
    \item Generates bulk of statistics where there is very little data. Optimal use of \textcolor{red}{CPU resources}?
\vspace{0.1cm}
    \item The same thing eats into resources for generating much-required MC statistics where the bulk of the data is. 
  \end{itemize}
\vspace{0.4cm}
     \item Similar problem for \textcolor{red}{$B_s \to KK\mu\mu$}, \textcolor{red}{$\Lambda_b \to pK\mumu$}, where a few resonances close to threshold strongly dominate the Dalitz plane.
\vspace{0.4cm}
     \item Generate something that is ``somewhat close'' to the Data, so that we maintain a \textcolor{red}{reasonable MC:Data ratio} over the entire \textcolor{red}{Dalitz} plane.
  \end{itemize}
}

%\frame{
%   \frametitle{Why a new generator?}
%  \begin{itemize}
%    \item Generic \textcolor{red}{$b \to X(\to h h) \mu \mu$ angular analyses}: $B_d \to K\pi\mu\mu$, $B_s \to KK\mu\mu$, $\Lambda_b \to pK\mu\mu$.
%\vspace{0.7cm}
%     \item Thrust is to go to wide $m(hh)$ window. 5-D phase-space:\\\textcolor{red}{$\phi \in \{\qsq, m_X, \ctl, \ctv, \chi\}$}. 
%\vspace{0.7cm}
%    \item  Mature \href{https://indico.cern.ch/event/407479/contribution/5/attachments/1141793/1635757/5D-Acceptance-Espen.pdf}{procedure} to get event-by-event $\epsilon(\phi)$. Generated MC first \textcolor{red}{reweighted} to \textcolor{red}{flat} in $\phi$.
%\vspace{0.7cm}
%    \item Need \textcolor{red}{high statistics MC}, especially at the PHSP edges in $\phi$.  
%  \end{itemize}
%}

%\frame{
%   \frametitle{Why a new generator? (cntd.)}
%  \begin{itemize}
%    \item Typically, {\tt PHSP} is used, but can be highly unoptimized for analysis needs.
%\vspace{0.1cm}
%    \item Example: for $B_d \to K\pi\mu\mu$, current analysis aims only till $m(K\pi) = 1530$~MeV. {\tt PHSP} is a waste: 
%  \end{itemize}
%\vspace{0.1cm}
%      \centering
%      \includegraphics[width=1.8in]{/home/hep/biplabd/cmtuser/analysis/test_XLL_Generator/plots/m_Kpi_KpiMuMu_PHSP.pdf}
%  \begin{itemize}
%\vspace{0.1cm}
%    \item Aim: generate spectra that reasonably follows data. We'll \textcolor{red}{reweight} this to \textcolor{red}{flat anyways}. 
%  \end{itemize}
%}

\frame{
   \frametitle{Issues at $|\cos \theta|\to 1$ in $B\to hh\mu\mu$}
  \begin{itemize}
     \item Acceptance parameterized by Legendre polynomials in the helicity angles.
     \item The Legendre polynomials in $x = \cos \theta$ are sensitive to to the boundaries at $x = \pm 1$.
     \item \textcolor{red}{Efficiencies} tend to be typically lowest here. Showing for $B_d\to K\pi\mu\mu$: 
  \end{itemize}
  \vspace{0.15cm}
   \begin{columns}
      \column{0.5\textwidth}
      \centering
      \includegraphics[width=2.3in]{figs/intro/espen_negative_efficiencies.pdf}
      \column{0.5\textwidth}
      \centering
      \includegraphics[width=2.3in]{figs/intro/espen_lowest_q2_ctl.pdf}
   \end{columns}
  \begin{itemize}
     \item Solution: \textcolor{red}{generate more} at \textcolor{red}{$|\cos \theta|\to 1$}. 
  \end{itemize}
}

\frame{
   \frametitle{Proposed $m(hh)$ cocktail spectra}
  \begin{itemize}
     \item \textcolor{red}{$f_{hh}$}: cocktail pdf out of simple {\tt Exp} and {\tt Gaus} functions. 
  \end{itemize}
\vspace{0.5cm}
   \begin{columns}
      \column{0.33\textwidth}
      \centering
      $m(K\pi)$:\\
      \includegraphics[width=1.6in]{/home/hep/biplabd/cmtuser/analysis/test_XLL_Generator/plots/m_Kpi.pdf}
      \column{0.33\textwidth}
      \centering
      $m(KK)$:\\
      \includegraphics[width=1.6in]{/home/hep/biplabd/cmtuser/analysis/test_XLL_Generator/plots/m_KK.pdf}
      \column{0.33\textwidth}
      \centering
      $m(pK)$:\\
      \includegraphics[width=1.6in]{/home/hep/biplabd/cmtuser/analysis/test_XLL_Generator/plots/m_pK.pdf}
   \end{columns}
}

\frame{
   \frametitle{Proposed $f_{\rm cth}$ pdf}
  \begin{itemize}
     \item \textcolor{red}{$f_{\rm cth}$}: generate more at $\cos \theta \to \pm 1$. 
  \end{itemize}
\vspace{0.5cm}
      \centering
      \includegraphics[width=2in]{/home/hep/biplabd/cmtuser/analysis/test_XLL_Generator/plots/cth.pdf}
}


\frame{
   \frametitle{The {\tt XLL} generator}
  \begin{itemize}
     \item Written for the generic $b \to X (\to hh)\ell \ell$ decay. Let $p_B$ and $p_h$ be the $B$ and $X$ breakup momentum.
\vspace{0.6cm}
     \item Existing \href{http://arxiv.org/abs/hep-ex/0409046}{\textcolor{blue}{{\tt FLATQ2}} generator} reweights {\tt PHSP} by \textcolor{blue}{$1/p_B$}. Used in $\babar$ $B \to X_u \ell \nu$ analyses. 
\vspace{0.6cm}
     \item \textcolor{red}{{\tt XLL}}: reweight {\tt PHSP} by  \textcolor{red}{$1/(p_B p_h)$}. Flat in $\qsq$-$m_X$. Now reweight by \textcolor{red}{$f_{\rm cth}$} and \textcolor{red}{$f_{hh}$}. 
     \vspace{0.6cm}
  \end{itemize}
}

\frame{
   \frametitle{The {\tt XLL} generator (contd.)}
  \begin{itemize}
   \item Weight $\textcolor{red}{w = f_{\rm cth} \times f_{hh} / (p_B p_h)}$. 
\vspace{0.6cm}
   \item Reweight the normalization integrals:\\ rate $\sim d\Phi |\mathcal{A}|^2\times\textcolor{red}{\displaystyle \frac{1}{w}}$
\vspace{0.6cm}
   \item Generator validated with the Sim group. See JIRA task \href{https://its.cern.ch/jira/browse/LHCBGAUSS-516}{\textcolor{blue}{here}}.
  \end{itemize}
}


\frame{
   \frametitle{The $1/(p_B p_h)$ reweighting}
  \begin{itemize}
     \item $B_d \to K\pi \mu \mu$ as an example.
  \end{itemize}
\vspace{0.6cm}
   \begin{columns}
      \column{0.5\textwidth}
      \centering
      PHSP:\\
      \includegraphics[width=2.4in]{/home/hep/biplabd/cmtuser/analysis/test_XLL_Generator/plots/m_Kpi_q2_PHSP.pdf}
      \column{0.5\textwidth}
      \centering
       PHSP re-wtd-ed:\\
      \includegraphics[width=2.4in]{/home/hep/biplabd/cmtuser/analysis/test_XLL_Generator/plots/m_Kpi_q2_PHSP_pBpH.pdf}
   \end{columns}
}


\frame{
   \frametitle{Summary}
  \begin{itemize}
    \item New generator specifically tuned to efficiently generate events for $b \to hh\mu\mu$ analyses. 
\vspace{0.6cm}
    \item {\tt PHSP} generated $m(hh)$ spectrum {\em very} different from Data. This is the main problem.
\vspace{0.6cm}
    \item New generator \textcolor{red}{{\tt XLL}} is pseudo-{\tt PHSP} with some sculpting to retain \textcolor{red}{reasonable Data:MC ratio} thru'out.
  \end{itemize}
}

