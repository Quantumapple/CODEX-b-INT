\section{Event Selection and $S/B$ Separation}


%\frame{
%   \vspace{2cm}
%   \centering
%\framebox{\parbox{\dimexpr\linewidth-2\fboxsep-2\fboxrule}{
%   {\Large Note: All results are very preliminary, and do not represent official $\babar$ results}}}
%
%}


\frame{
  \frametitle{Semi-leptonic $B$ decays}
       \vspace{-0.2cm}
      \begin{itemize}
          \item \textcolor{red}{Cleanest} source of \textcolor{red}{$\BB$ pairs} for studying SL $B$ decays: \\ \vspace{0.05cm} \hspace{1.5cm}$\textcolor{red}{e^+ e^-} \to \ups \overset{{\Tiny \textcolor{red}{> 96\%}}}{\to} \textcolor{red}{\BB}$, $\sqrts = \textcolor{red}{10.58}$~GeV
          \vspace{0.15cm}
          \item PEP-II collider and \textcolor{red}{$\babar$} detector: asymmetric $e^+ e^-$ $B$-factory at SLAC. %Data-taking between 1999-2008, but analyses ongoing. 
          %\item \textcolor{red}{$\babar$}: multipurpose detector for PEP-II $e^+ e^-$ asymmetric storage ring at SLAC. 
      \end{itemize}
\vspace{0.05cm}
  \begin{columns}
    \column{0.55\textwidth}
      \begin{itemize}
          \item \textcolor{red}{Full} ``on-peak'' $\babar$ dataset: \textcolor{red}{$\sim 470$~M $\BB$} pairs
          \vspace{0.3cm} 
           \item Looking at $\Bbar \to \textcolor{red}{X_{\{c,u\}}} \ell^- \barnuell$, $\textcolor{red}{X_{\{c,u\}}} \in \{\textcolor{red}{D, D^\ast, \rho} \}$
          \vspace{0.3cm} 
          \item \textcolor{red}{Tagged} analysis w/ setup similar to recent $\Bbar \to D^{(\ast)} \tau^- \bar{\nu}_\tau$ analysis ({\small PRL 109, 101802 (2012)}).
      \end{itemize}

    \column{0.45\textwidth}
       \vspace{-0.3cm}
       \centering
       { \textcolor{red}{$B^- \to \rhoz(\to \pip \pim) \mu^- \nu_{\mbox{\tiny miss}}$}}:\\ \vspace{0.3cm}
       \includegraphics[width=2in]{figs/data_anal/dpf13_babar_rho0lnu_event.jpg}
  \end{columns}
}




\frame{
   \frametitle{The BReco technique at $B$-factories}% and untagged/tagged measurments}
   \vspace{-0.1cm}
         \begin{itemize}
            \item In $e^+ e^- \to \FourS \to \bsig\btag$, \textcolor{red}{full hadronic reconstruction} of $\btag$ (\textcolor{red}{BReco}) in $\sim 3000$ modes. 
         \end{itemize}
   \begin{columns}
      \column{0.5\textwidth}
         \vspace{-0.1cm}
         \begin{itemize}
            %\item In $e^+ e^- \to \FourS \to \bsig\btag$, \textcolor{red}{full hadronic reconstruction} of $\btag$ (\textcolor{red}{BReco}) in $\sim 3000$ modes
            \item \textcolor{blue}{$\btag\to$} \textcolor{DarkGreen}{$Y_{\scriptsize \text{had}}$ ($\pi$'s and $K$'s)} + \textcolor{DeepPink}{charm ``Seed''}.
            \vspace{0.2cm}
            \item A \textcolor{red}{single missing $\nu$}: reconstructed as $p_{\scriptsize \text{miss}}$ via kinematic fit. 
            %\item Similar to $e^+ e^- \to \psi(3770) \to \dtag \dsig$ in CLEO-c/BESIII
            %\vspace{0.5cm}
            %\item Can't go to $\bsig$ RF w/o BReco
         \end{itemize}
      \column{0.55\textwidth}
         %\centering
         \includegraphics[width=2.6in]{figs/intro/breco.pdf}
   \end{columns}
   \vspace{0.3cm} 

   \begin{itemize}
       \item \textcolor{blue}{Untagged} analyses: can't go to $\bsig$ RF; \textcolor{blue}{can't measure $\{\qsq, \ctl, \ctv, \chi\}$} directly.
       %\vspace{0.3cm} 
       %\item Similar techniques at \textcolor{red}{charm}-factories \textcolor{red}{$e^+ e^- \to \psi(3770) \to \dsig \dtag$}.  
   \end{itemize}
}



\frame{
   \frametitle{The BReco technique at $B$-factories (cntd.)}% and untagged/tagged measurments}
   \begin{columns}
      \column{0.5\textwidth}
         \centering
         \includegraphics[width=2.4in]{figs/intro/rho0_resolution_poster.pdf} \\ {\em Percent} level \textcolor{red}{resolution} due to BReco
      \column{0.5\textwidth}
         \begin{itemize}
            \item Very \textcolor{red}{clean} $\bsig$ sample. 
            \vspace{0.5cm}
            \item \textcolor{red}{Fantastic resolution} in reconstruction of angular variables  
         \end{itemize}
   \end{columns}

    \vspace{0.6cm}

    \begin{itemize}
        \item Gives $B$-factories an advantage over LHCb for neutrinos.
    \end{itemize}
}



%\frame{
%   \frametitle{The BReco technique at $B$-factories (cntd.)}% and untagged/tagged measurments}
%   \vspace{-0.1cm}
%   \begin{columns}
%      \column{0.6\textwidth}
%         \vspace{-0.1cm}
%         \begin{itemize}
%            \item In $e^+ e^- \to \FourS \to \bsig\btag$, \textcolor{red}{full hadronic reconstruction} of $\btag$ (\textcolor{red}{BReco}) in $\sim 3000$ modes
%            \vspace{0.2cm}
%            \item A \textcolor{red}{single missing $\nu$} and BReco together allows direct \textcolor{red}{access} to all \textcolor{red}{final state momenta}
            %\item Similar to $e^+ e^- \to \psi(3770) \to \dtag \dsig$ in CLEO-c/BESIII
            %\vspace{0.5cm}
            %\item Can't go to $\bsig$ RF w/o BReco
%         \end{itemize}
%      \column{0.55\textwidth}
         %\centering
%         \includegraphics[width=2in]{figs/intro/breco.pdf}
%   \end{columns}
%   \vspace{0.3cm} 
 
   %\pause
%   \begin{columns}
%      \column{0.4\textwidth}
%         \centering
%         \includegraphics[width=1.8in]{figs/intro/rho0_resolution_poster.pdf} \\ {\scriptsize {\em Percent} level \textcolor{red}{resolution} due to BReco}
%      \column{0.6\textwidth}
%         \begin{itemize}
%            \item Very \textcolor{red}{clean} $\bsig$ sample, with \textcolor{red}{fantastic resolution} in reconstruction of angular variables  
%            \vspace{0.1cm}
%            \item Gives $B$-factories a distinct advantage over LHCb for neutrinos.
            %\item \textcolor{blue}{Untagged} $\Rightarrow$ \textcolor{blue}{no access} to \textcolor{blue}{$\bsig$ RF}. Kinematic variables not measured!
            %\pause
            %\item Note: angular analysis/motivations quite similar to $\Bbar_d \to \overline{K^\ast} \ell^- \ell^+$, except SM contribution is large now and we ``detect'' both leptons indirectly using BReco.
            %\vspace{0.3cm}
            %\item Technique meant for signal-side neutrino(s). Optimal for a {\em single} missing neutrino.
            %\item Low efficiency ($\sim \mathcal{O}(0.1\%)$), but expected to become the de facto method at Belle~II. 
            %\vspace{0.3cm}
            %\item {\em BReco not possible at a hadronic collider}. LHCb can do $B\to V\ell^-\ell^+$, but $\Bbar \to V \ell^- \barnuell$ is difficult. 

%         \end{itemize}
%   \end{columns}
%}





\frame{
   \frametitle{Event Selection} 


     \begin{itemize}
        \item $\FourS \to \textcolor{red}{\btag X_{\{c,u\}} \ell} \;(\nu_{\scriptsize \text{miss}})$ \textcolor{red}{fully reconstructed}. No extra charged tracks. Flavor and charge constraints automatically satisfied. 
        \vspace{0.3cm}
          \item Two \textcolor{blue}{tag-side} discriminating variables:
      \end{itemize}
         \vspace{-0.1cm}
      \begin{align}
         \textcolor{blue}{\de} &= E_{\mbox{{\scriptsize tag}}} - E_{\mbox{{\scriptsize beam}}} \;(\sim \textcolor{blue}{0} \; \text{for signal}) \nonumber \\ 
         \textcolor{blue}{\mes} &=\sqrt{E^2_{\mbox{{\scriptsize beam}}} - |\vec{p}_{\mbox{{\scriptsize tag}}}|^2} \;(\sim \textcolor{blue}{m_B} \; \text{for signal})\nonumber
         %\textcolor{blue}{\de} &= E^\ast_B - \sqrts/2 \;(\sim \textcolor{blue}{0} \; \text{for signal}) \nonumber \\ 
         %\textcolor{blue}{\mes} &=\sqrt{s/4 - |\vec{p}^\ast_B|^2} \;(\sim \textcolor{blue}{m_B} \; \text{for signal})\nonumber
       \end{align}
         \vspace{-0.15cm}
     \begin{itemize}
        \item Since $p_\nu \equiv p_{\scriptsize \text{miss}}$, \textcolor{red}{signal}-fit variable is:\\
        \begin{equation} 
            \textcolor{red}{U = E^\ast_\nu - |\vec{p}^\ast_\nu|}\;\;(\text{in } B\text{ RF}) \nonumber
        \end{equation} 
        \item Skim-level selection: \textcolor{blue}{$\mes > 5.27$}~GeV, \textcolor{blue}{$|\de| < 72$}~MeV and \textcolor{red}{$|U| < 0.4$}~GeV. Already a well-constrained system. 
      \end{itemize}
}

\frame{
    \frametitle{Other $S/B$ Discriminating Variables}

   \begin{columns}
      \column{0.7\textwidth}
          \normalsize
          \begin{itemize}
              \item Require \textcolor{red}{good photons} to deposit $\geq \textcolor{red}{50}$~MeV/cluster in the calorimeter
              \vspace{0.2cm}
              \item \textcolor{red}{$\eext$} is sum of energies of \textcolor{red}{extra good photons} not used in event reconstruction (bkgd. in the EMC)
          \end{itemize}
      \column{0.3\textwidth}
       \centering
       \includegraphics[width=1.4in]{figs/data_anal/zurich_talk_eex.pdf} 
   \end{columns}
   \begin{columns}
      \column{0.7\textwidth}
          \begin{itemize}
          \normalsize
              \item \textcolor{red}{$\Delta \thrustA$}: angle between $\btag$ and Rest-Of-Event \textcolor{red}{thrust axes} (direction that maximizes the longitudinal momentum) 
              \vspace{0.2cm}
              \item \textcolor{red}{$\BB$} events are \textcolor{red}{isotropic}, while \textcolor{blue}{$\qqb$} (\textcolor{blue}{continuum}) events are \textcolor{blue}{jet-like}. 
          \end{itemize}
      \column{0.3\textwidth}
       \includegraphics[width=1.4in]{figs/data_anal/zurich_talk_cosT.pdf} 
   \end{columns}
}

\frame{
   \frametitle{Kinematic Fitting}
    \begin{itemize}
      \item ``Standard'' fit (\textcolor{red}{Fit~I}):
      \vspace{0.2cm}
      \begin{itemize}
        \normalsize
         \item except for $\rho$, \textcolor{red}{$X$} is \textcolor{red}{mass-constrained}
         \vspace{0.2cm}
         \item except for $D$, \textcolor{red}{$X$} vertex is \textcolor{red}{beam-spot}-constrained
         \vspace{0.2cm}
         \item both \textcolor{red}{$\bsig$} and \textcolor{red}{$\btag$} are \textcolor{red}{mass-constrained}
         \vspace{0.2cm}
         \item the \textcolor{red}{$\ups$} candidate vertex is constrained to the \textcolor{red}{beam-spot}
         \vspace{0.2cm}
         \item The signal variable \textcolor{red}{$U$} comes from \textcolor{red}{Fit~I}
      \end{itemize}
      \vspace{0.4cm}
    \item Additional fit (\textcolor{red}{Fit~II}):
      \begin{itemize}
        \normalsize
        \vspace{0.2cm}
        \item Additionally, \textcolor{red}{$\nu$} \textcolor{red}{mass-constrained} as well
        \vspace{0.2cm}
        \item All kinematic variables \textcolor{red}{other than $U$} come from \textcolor{red}{Fit~II}.
      \end{itemize}
   \end{itemize}
}

\frame{
   \frametitle{Event selection: $\Bbar \to D^\ast \ellm \barnuell$}
    \begin{itemize}
       \item Require \textcolor{red}{$\deltam \equiv m(D^\ast) -m(D)$} less than $4\sigma$ around nominal value, where $\sigma$ is the expected resolution.
        \vspace{0.5cm}
       \item $\Bbar \to \textcolor{red}{D^\ast (\to D \pi)}\; \ellm \barnuell$ is \textcolor{red}{{\em very} clean} after this selection.
        \vspace{0.5cm}
       \item Carefully select only \textcolor{red}{six clean modes} with \textcolor{red}{$\sim 2\%$} background after CL cut from \textcolor{red}{Fit~II} (\textcolor{red}{$\nu$ mass-constrained} in Kin. Fit). 
        \vspace{0.5cm}
       \item \textcolor{red}{{\em No background-subtraction needed!}}
    \end{itemize}
}

\frame{
   \frametitle{Event selection: $\Bbar \to D^\ast \ellm \barnuell$ (cntd.)}
\vspace{-0.1cm}
    \begin{itemize} 
      \item Expected yield and $B/S$ estimates using MC, w/ $|U|<50$~MeV cut:
    \end{itemize} 
\vspace{0.1cm}

  \begin{columns}
    \column{0.5\textwidth}
     \centering
     \underline{$D^{\ast+} \to \Dz \pip$}
\vspace{0.2cm}

\begin{tabular}{l c c }
  $\Dz$ Mode & $S$ & $B/S (10^{-2})$ \\ \hline
 $K^-\pip$ & 825 & 1.1\\ \hline
 $K^-\pip \piz$ & 1434 & 1.4 \\ \hline
 $K^-\pip \pim \pip$ & 943 & 1.3\\ \hline
\end{tabular}
    \column{0.5\textwidth}
     \centering
     \underline{$D^{\ast 0} \to \Dz \piz$}
\vspace{0.2cm}

\begin{tabular}{l c c }
  $\Dz$ Mode & $S$ & $B/S (10^{-2})$ \\ \hline
 $K^-\pip$ & 873 & 2.5\\ \hline
 $K^-\pip \piz$ & 1193 & 2.6\\ \hline
 $K^-\pip \pim \pip$ & 825 & 2.6\\ \hline
\end{tabular}
  \end{columns}
 \vspace{0.3cm}
  \begin{columns}
    \column{0.5\textwidth}
    \centering
    \includegraphics[width=2.2in]{figs/data_anal/bdslnu.pdf} 
    \column{0.5\textwidth}
    \begin{itemize} 
      \item Remarkably clean dataset! 
       \vspace{0.4cm}
    \end{itemize} 
     \centering
      \fcolorbox{red}{yellow}{$\sim 6000$ events w/ $\sim 2\%$ bkgd.}
   \end{columns}
}


\frame{
   \frametitle{Event selection: $\Bbar \to D \ellm \barnuell$}

     \vspace{0.3cm}
  \begin{columns}
    \column{0.45\textwidth}

     \vspace{-0.6cm}
    \begin{itemize} 
      \item \textcolor{red}{Five} clean enough modes: 
     \vspace{0.1cm}
    \begin{itemize}
       \normalsize
        \item $\Dz \to K^-\pip$  
        \vspace{0.05cm}
        \item $\Dz \to K^- \pip \piz$
        \vspace{0.05cm}
        \item $\Dz \to K^- \pip \pim \pip $
        \vspace{0.05cm}
        \item $\Dp \to K^- \pip \pip $
        \vspace{0.05cm}
        \item $\Dp \to K^- \pip \pip \piz $
    \end{itemize}
     \vspace{0.8cm}
     \item \textcolor{red}{Loose cuts}: $\eext < 0.8$~GeV and CL from Fit~I $\geq 10^{-8}$ 
    \end{itemize} 

    \column{0.6\textwidth}
    \vspace{-1.4cm}
    \begin{itemize} 
      \item Main background: feed-down from $D^\ast$ 
    \end{itemize} 
    \begin{center}
    \includegraphics[width=2.5in]{figs/data_anal/btodlnu_stack_2.pdf} \\ \vspace{-1.4cm} \hspace{3cm} {\small {\tt [Not approved]}} 
    \end{center}

  \end{columns}

      \vspace{0.2cm}
    \begin{itemize} 
      \item Note the \textcolor{red}{smooth} nature of the \textcolor{red}{background shape} under the signal  
      \vspace{0.2cm}
      \item Caveat: ``out-of-the-box'' existing generic $\babar$ MC. MC {\em not} fitted to Data.
    \end{itemize} 
}

\frame{
   \frametitle{Event selection: $\Bbar \to \rhoz \ellm \barnuell$}
    \begin{itemize} 
      \item For \textcolor{red}{charmless} modes ($\{\pi, \rho,\omega,...\}$), the \textcolor{red}{$e/\mu$} samples have to be processed \textcolor{red}{separately}
      \vspace{0.45cm}
      \item \textcolor{red}{$\pi \leftrightarrow \mu$ misId}: bkgd. characteristics very different between $e/\mu$ 
      \vspace{0.45cm}
      \item \textcolor{red}{$\mu$} sample has more $\qqb$ \textcolor{red}{continuum} bkgd. that hadronizes to multi-charmless final states and \textcolor{red}{nothing missing. $|\vec{p}_{\text{\scriptsize miss}}| > 0.3$} GeV.
      \vspace{0.45cm}
      \item Tighter \textcolor{red}{$|\cosT|$} cuts for \textcolor{red}{$\mu$} sample for \textcolor{red}{continuum suppression}.
      \vspace{0.45cm}
      \item \textcolor{red}{$e$ signal} shape has a long \textcolor{red}{tail} at high $U$ due to larger \textcolor{red}{brem.} 
      \vspace{0.45cm}
      \item \textcolor{red}{$M(\pi \pi) \in[0.67,0.87]$}~GeV. {\em Assume} no $S$-wave (checks done using $\piz \piz \ell \nu$). 
    \end{itemize} 
}

\frame{
   \frametitle{Event selection: $\Bbar \to \rhoz \ellm \barnuell$ (cntd.)}

    \begin{center}
    \includegraphics[width=2.5in]{figs/data_anal/rholnu_stack.pdf} \\ \vspace{-1.7cm} \hspace{3cm} {\small {\tt [Not approved]}}
    \end{center}
    \vspace{1cm}
    \begin{itemize} 
      \item Note the \textcolor{red}{smooth} nature of the \textcolor{red}{background shape} under the signal  
      \vspace{0.2cm}
      \item Caveat: ``out-of-the-box'' existing generic $\babar$ MC. MC {\em not} fitted to Data.
    \end{itemize} 
}


\frame{
   \frametitle{$S/B$ separation fits for $X\in \{D,\rhoz\}$}

    \begin{itemize} 
      \item From MC: smooth \textcolor{blue}{background}, parameterize as a smooth Gaussian \textcolor{blue}{tail}.
      \vspace{0.1cm}
      \item \textcolor{red}{Signal} lineshape: sum of \textcolor{red}{bifurcated Gaussians} centered around \textcolor{red}{$U\sim 0$}. 
      \vspace{0.1cm}
      \item \textcolor{blue}{Start parameters} from \textcolor{blue}{MC}, but kept \textcolor{red}{floating} during fits to \textcolor{red}{Data}.
      \vspace{0.1cm}
      \item Background is phase-space dependent. {\em Perform \textcolor{red}{fits} in \textcolor{red}{small phase-space volumes}}.
      \vspace{0.1cm}
      \item For $i^{th}$ event choose $N_c$ ``\textcolor{red}{closest-neighbor}'' events using ad hoc \textcolor{red}{metric}: 
    \begin{equation}
       d^2_{ij} = \sum_\alpha \left(\frac{\phi^\alpha_i - \phi^\alpha_j}{r^\alpha}\right)^2\;,\; \phi^\alpha\in \{\qsq, \ctl,\ctv,\chi\} \nonumber
    \end{equation}
     where $r^\alpha$ is the range of $\phi^\alpha$.
    \end{itemize} 

}



\frame{
   \frametitle{$S/B$ separation fits for $X\in \{D,\rhoz\}$ (cntd.)}

    \begin{itemize} 
       \item Fit to set of $N_c+1$ events in the discriminating variable $U$. Obtain \textcolor{red}{signal} ($\textcolor{red}{\mathcal{S}}_i(U)$) and \textcolor{blue}{background} ($\textcolor{blue}{\mathcal{B}}_i(U)$) functions.
      \vspace{0.2cm}
       \item Assign a quality-factor $Q_i$, probability for the event to be signal: 
    \end{itemize} 
     \begin{center}
    \framebox{ \textcolor{red}{$Q_i = \displaystyle \frac{\mathcal{S}_i(U_i)}{\mathcal{S}_i(U_i) + \mathcal{B}_i(U_i)}$} }
     \end{center}
      \vspace{0.2cm}
    \begin{itemize} 
       \item Use \textcolor{red}{$ Q_i$} as a \textcolor{red}{weight} for the event in subsequent angular fits.
      \vspace{0.2cm}
      %\item $_s\mathcal{P}lot$ technique: \textcolor{red}{$U$ (discriminating)} (\textcolor{blue}{$\phi$}) as \textcolor{red}{discriminating} (\textcolor{blue}{control}) variables.
      %\vspace{0.1cm}
       \item Main utility: tracks all \textcolor{red}{multi-dimensional angular correlations} present in the signal-component of the Data 
      \vspace{0.2cm}
       \item Refs: {\small JINST 4 P10003 (2009); PRC 80, 025202 (2010); PRC 80, 065208 (2009); PRC 80, 065209 (2009); PRC 80, 045213 (2009).}
    \end{itemize} 
}

\frame{
   \frametitle{$S/B$ separation w/ $Q$-values:  $\Bbar \to D \ellm \barnuell$}

  \begin{columns}
    \column{0.45\textwidth}

     \begin{itemize}
         \item ``\textcolor{red}{mock-Data}'' using \textcolor{red}{MC}: representative of Data in both content and statistics.  
         \vspace{0.5cm}
         \item Each $D$ and $\ell$ mode must be processed independently (different background characteristics)
         \vspace{0.5cm}
         \item From event-by-event fits:\\
         \;\; \textcolor{red}{signal} weight: \textcolor{red}{$Q_i$}\\
         \;\; \textcolor{blue}{background} weight: \textcolor{blue}{$(1-Q_i)$}
     \end{itemize}

    \column{0.6\textwidth}
    \centering
    \underline{$\Dz \to K^- \pip \piz$}:
    \vspace{0.25cm}
    \includegraphics[width=2.65in]{figs/data_anal/sig_bkgd_dist_cutoff3_tweak70_tweak_sig95_nc50_d0_mode2_U.pdf} 
  \end{columns}

}


\frame{
   \frametitle{$Q$-value weighted distributions:  $\Dz \to K^- \pip \piz$}
     \vspace{-0.3cm}
     \begin{itemize}
         \item Apply \textcolor{red}{$|U| < 0.05$}~GeV cut to \textcolor{red}{remove sidebands} now (esp. brem tail).
         \vspace{0.1cm}
         \item \textcolor{red}{1-D projections} with the \textcolor{red}{$Q$-value weights}: 
         \vspace{0.15cm}
     \end{itemize}

  \begin{columns}
    \column{0.5\textwidth}
    \centering
    \includegraphics[width=2.4in]{figs/data_anal/sig_bkgd_dist_cutoff3_tweak70_tweak_sig95_nc50_d0_mode2_q2.pdf} 
    \column{0.5\textwidth}
    \centering
    \includegraphics[width=2.4in]{figs/data_anal/sig_bkgd_dist_cutoff3_tweak70_tweak_sig95_nc50_d0_mode2_ctl.pdf} 
  \end{columns}
}



\frame{
   \frametitle{$S/B$ separation w/ $Q$-values:  $\Bbar \to \rhoz \ellm \barnuell$}

  \begin{columns}
    \column{0.45\textwidth}

     \begin{itemize}
       \item \textcolor{red}{$\rhoz \to \pip \pim$}: much more challenging.
         \vspace{0.35cm}
       \item \textcolor{red}{Low statistics}, \textcolor{red}{4-D} problem, should \textcolor{red}{$M(\pi \pi)$} be a discriminating variable as well?
         \vspace{0.35cm}
       \item \textcolor{red}{Simplify} with $M(\pi\pi)\in [0.67,0.87]$~GeV selection and treat as \textcolor{red}{2-D phase-space} in \textcolor{red}{$\{\qsq, \ctl\}$} for $S/B$ separation.
     \end{itemize}
    \column{0.6\textwidth}
      \centering
    \underline{$\rhoz \to \pip \pim$}:
    \vspace{0.25cm}
       \includegraphics[width=2.65in]{figs/data_anal/sig_bkgd_rholnu_U.pdf}
  \end{columns}
}


\frame{
   \frametitle{$Q$-value weighted distributions:  $\rhoz \to \pip \pim$}
     \vspace{-0.3cm}
     \begin{itemize}
         \item Apply \textcolor{red}{$|U| < 0.06$}~GeV selection to \textcolor{red}{remove sidebands} now.
         \vspace{0.1cm}
         \item Sample \textcolor{red}{1-D projections} with the \textcolor{red}{$Q$-value weights}: 
         \vspace{0.15cm}
     \end{itemize}

  \begin{columns}
    \column{0.5\textwidth}
    \centering
    \includegraphics[width=2.4in]{figs/data_anal/sig_bkgd_rholnu_q2.pdf} 
    \column{0.5\textwidth}
    \centering
    \includegraphics[width=2.4in]{figs/data_anal/sig_bkgd_rholnu_ctl.pdf} 
  \end{columns}
}


\frame{
   \frametitle{$Q$-value method: some remarks}

     \begin{itemize}
         %\item This is a \textcolor{red}{probabilistic method} -- requires {\em reasonable} statistics.  
         %\vspace{0.5cm}
         \item Can't handle peaking component. \textcolor{red}{$\mathcal{S}$} and \textcolor{red}{$\mathcal{B}$} pdf's assumed \textcolor{red}{disjoint}. 
         \vspace{0.5cm}
         \item \textcolor{red}{No interference} assumed between $S$ and $B$ (eg. $S$-$P$-wave intereference).
         \vspace{0.5cm}
         \item \textcolor{red}{Similar} methods already adopted by \textcolor{red}{FOCUS/CLEO-c/BES~III} for semi-leptonic $D$ decays (see {\small arXiv:1302.0227})
         \vspace{0.5cm}
         \item \textcolor{red}{Systematics} mainly based on \textcolor{red}{varying} phase-space volume (or number of closest-neighbors \textcolor{red}{$N_c$}).
         %\vspace{0.5cm}
         %\item Probabilistic method -- requires ``reasonable'' statistics. %Application on $\Bbar \to \rhoz \ellm \barnuell$ slightly tricky.
     \end{itemize}
}



