% $Id: title-LHCb-INT.tex  $
% ===============================================================================
% Purpose: LHCb-INT Note title page template
% Author: P. Koppenburg
% Created on: 2015-05-18
% ===============================================================================

%%%%%%%%%%%%%%%%%%%%%%%%%
%%%%%  TITLE PAGE  %%%%%%
%%%%%%%%%%%%%%%%%%%%%%%%%
\begin{titlepage}

% Header ---------------------------------------------------
\vspace*{-1.5cm}

\noindent
\begin{tabular*}{\linewidth}{lc@{\extracolsep{\fill}}r@{\extracolsep{0pt}}}
\ifthenelse{\boolean{pdflatex}}% Logo format choice
{\vspace*{-1.2cm}\mbox{\!\!\!\includegraphics[width=.14\textwidth]{figs/lhcb-logo.pdf}} & &}%
{\vspace*{-1.2cm}\mbox{\!\!\!\includegraphics[width=.12\textwidth]{figs/lhcb-logo.eps}} & &}
 \\
 & & LHCb-INT-2018-XXX \\  % ID 
 & & \today \\ % Date - Can also hardwire e.g.: 23 March 2010
 & & \\
\hline
\end{tabular*}

\vspace*{4.0cm}

% Title --------------------------------------------------
{\normalfont\bfseries\boldmath\huge
\begin{center}
% DO NOT EDIT HERE. Instead edit macro in main.tex to keep metadata correct
  \papertitle
\end{center}
\begin{center}
    { \large CERN summer student report }
\end{center}
}

\vspace*{1.5cm}

% Authors -------------------------------------------------
\begin{center}
% If changing to list here, make pdfauthors in main.tex a comma
% separated list with the same names. Otherwise metadata in file will be wrong.
%Biplab Dey$^1$, Jongho Lee$^2$, ...
Jongho Lee$^1$ %...
\bigskip\\
{\normalfont\itshape\footnotesize
%$ ^1$CCNU\\
$ ^1$CERN \\
}
\vspace{0.5cm}
1st supervisor: B. Dey \\
\vspace{0.2cm}
2nd supervisor: V. Coco

\end{center}

\vspace{\fill}

% Abstract -----------------------------------------------
\begin{abstract}
  \noindent
  %Include abstract here.
  CODEX-b is a newly proposed detector~\cite{Gligorov:2017nwh} to be housed inside the existing LHCb cavern that will search for weakly interacting long-lived particles, predicted in many extensions of the Standard Model. A critical component in the physics reach studies is good understanding of the expected background rates inside the cavern. As a CERN summer student during June-August, 2018, I participated in a campaign to measure the background rate in the UX85A cavern during Run~2 $pp$ collision data-taking. The measurements were performed at various positions and different configurations on the D3 platform in UXA just behind the existing concrete shield wall. The campaign was very successful with over 50,000 recorded triggers. In addition, I also developed a simulation framework for CODEX-b and the measurement setup, using a {\tt ROOT} based Detector Description package called {\tt DD4Hep}, that will be used by the LHC experiments in the Upgrade era. Preliminary results not officially approved by the LHCb collaboration, are presented here.
\end{abstract}

\vspace*{2.0cm}
\vspace{\fill}

\end{titlepage}


\pagestyle{empty}  % no page number for the title 

%%%%%%%%%%%%%%%%%%%%%%%%%%%%%%%%
%%%%%  EOD OF TITLE PAGE  %%%%%%
%%%%%%%%%%%%%%%%%%%%%%%%%%%%%%%%

%  empty page follows the title page ----
%\newpage
%\setcounter{page}{2}
%\mbox{~}
%
%\cleardoublepage
