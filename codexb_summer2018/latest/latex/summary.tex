\section{Summary}
\label{sec:Summary}

It was a very successful measurement campaign at D3 platform. 
We measured hit rates of mip based on 2x fold coincidence trigger using Herschel detector and scope.
We could also manage run and data by remote connect to scope.
The average hit rates in the stable beam condition is much higher than the average hit rates of pure background.
Based on this result, we can ignore pure background.
The background rate just behind the concrete shield wall around 0.5~Hz over 900~$cm^{2}$ size scintillators.
Also the hit rates have $\eta$ dependence when moving far from the impact point.
The D3 racks behave like a shield from the P5 and P6 results but it is difficult to simulate because of complicated structure.

About the simulation, DD4hep had been used to design CODEX-b and backgrond measurement campaign.
Built a hierachy system to implement a bundle of 6 silicon layers (these layers are planed to change to RPC layers) and a triplet bundle.
Reminding that CODEX-b geometry is a final version. 
Detail information about components of designed CODEX-b using DD4hep is following. 
The layer runs as a tracker which measure hit position of particles and the size is 10x10~$m^{2}$ with 2~cm thickness.
Using similar hierachy system, background measurement campaign geometry had been made.
The size of scintillator plate is 30x30~$cm^{2}$ and 2~cm thickness.
The distance of them is 2~cm.
The material of scintillator is the same as the Herschel detector.
Veto cone and concrete shield wall were generated using DD4hep.

All geometries were tested with $\mu$ particle gun with and without concrete shield wall.
There were hits on the layers without shield wall and checked particles hit positions and deposit energy.
Tested with minimum bias events generated from standalone Gauss showed hits on the layers without concrete wall.
When the wall was existed, there was no hit on the layers.
These results indicate that every part in the simulation worked properly.

The future plan is working on more efficient MC generation in Gauss with generator cuts and optimizing the simulation environments.





