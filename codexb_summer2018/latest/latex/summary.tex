\section{Summary}
\label{sec:Summary}

In summary, I participated in a very successful measurement campaign and we obtained the data we needed. We measured hit rates of charged mip's based on 2-fold coincidence trigger using elements from the HeRSCheL detector and scope. We also managed to take data efficiently by remote connection to the scope. The average hit rate under the stable beam condition is much higher than the average hit rates of pure ambient background. Based on this result, we can ignore pure background. The background rate just behind the concrete shield wall around 0.5~Hz over 900~$cm^{2}$ size scintillators. Also the hit rates have $\eta$ dependence when moving further downstream, from the impact point. The D3 racks behave like a shield from the P5 and P6 results but it is difficult to simulate due to the complicated material.

For the simulation, I used {\tt DD4hep} to design CODEX-b and backgrond measurement geometries and built a hierachy system to implement a bundle of 6 silicon layers (these layers are planed to change to RPC layers) and a triplet bundle, per station. I note here that this CODEX-b geometry is not final, but will be further developed, using the current setup. Each layer is simulated as a Silicon tracker that records the hit position of particles; its size is $10\times10$~m$^{2}$ with 2~cm thickness. Using a similar hierachy system, the geometry for the background measurement campaign geometry been built. The size of scintillator plate is $30\times30$~cm$^{2}$ and 2~cm thickness. The material of scintillator is the same as the Herschel detector.
The proposed veto cone and concrete shield wall were also generated using {\tt DD4hep}.

All geometries were tested with $\mu$ particle gun with and without concrete shield wall. WE found hits on the layers without shield wall and checked the hit positions and deposited energies. We also tested with minimum bias events generated from {\tt Gauss} and found hits on the layers without the concrete wall. When the shield wall as reinstated, there was no hits on the layers, indicating that simulation works as expected.

The future plan is to develop more efficient MOnte Carlo generation in {\tt Gauss} with optimized generator cuts and large enough statitics, to validate the measurement data.





