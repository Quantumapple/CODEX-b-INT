\section{Simulation}
\label{sec:Simulation}

\subsection{Detector Description for High Energy Physics}
I used Detector Description for High Energy Physics (DD4hep) standalone version.
DD4hep is a software framework to provide overall detector description for experiments.
It offers a consistent description through a single source of detector information for simulation, reconstruction, analysis, etc.
I learn how to make geometry: layer, station, super staion, envelope (hierachy).
I define materials for our detector and CODEX-b geometry such as concrete, Herschel detector.
Layer consists of silicon, station consists of aluminum. 
There is a veto cone with two lead and one silicon. 
Also just in front of CODEX-b, concrete wall exists to veto muons.
First of all, using muon particle gun with high energy, test our geometry.
And then using HepMC to generate pp collisions and do the same process as muon particle gun.
I made hierachy system to build CODEX-b (envelop, super station, station, layer).
I could check energy deposits and positions of CODEX-b hits.


\subsection{Simulation geometry}
To make coincidence setup with test-bench, I made two Herschel plates with the same positions where all equipment have set.
Two plates with 30 x 30~$cm^{2}$ size, 2~cm thickness.
There is a concrete wall in front of scintillators.
3~m thickness to suppress particles from pp collisions.
Roughly in 1000 events, it has hits on scinitillators 4 - 9 events.
There is a proposed veto cone. It consists of two lead absorbers and one silicon tracker.
There is also concrete wall which blocks radiations (or particles) to reach CODEX-b box. 
It has 3.2~m thickness.

\begin{figure}[h]
\begin{center}
  \begin{tabular}[t]{cc}
    \includegraphics[width=0.5\textwidth]{figs/INT/CODEXbBigGeo.pdf} &
    \includegraphics[width=0.5\textwidth]{figs/INT/ZoomVersion.pdf}
  \end{tabular}
\end{center}
\caption{
    Wide view of CODEX-b simulation geometry
}
\end{figure}

\subsection{Simulation status}

We designed two different detectors based on similar geometry, one is the CODEX-b and the other is scintillator.
Both were tested with $\mu$ particle gun with 1\tev and the minimum bias events generated from the standalone Gauss. 

\begin{figure}[h]
\centering
    \includegraphics[width=12cm]{figs/INT/Minbias.pdf}
\caption{
    Validation of the CODEX-b simulation by removing the concrete shield wall with minimum bias events 
}
\end{figure}

\begin{figure}[h]
\centering
    \includegraphics[width=12cm]{figs/INT/Scint.pdf}
\caption{
    Test of the scintialltor configuration with minimum bias events
}
\end{figure}
