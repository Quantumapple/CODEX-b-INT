\section{Simulation}
\label{sec:Simulation}

\subsection{Detector Description for High Energy Physics}
We used Detector Description for High Energy Physics (DD4hep) standalone version \href{https://dd4hep.web.cern.ch/dd4hep/}{[2]}.
DD4hep is a software framework to provide overall detector description for experiments.
It offers a consistent description through a single source of detector information for simulation, reconstruction, analysis, etc.
Additionally, DD4hep being developed for high luminosity large hadron collider (HL-LHC) detector simulation.
During the internship, we built the geometry of CODEX-b constructing hierachy system.
We designed concrete shield wall to block particles from particle gun or MC and herschel detector since we used as a scintillator for our measurement.
For validation $\mu$ particle gun and minbias event had been used.
We also checked energy deposits and positions of CODEX-b hits. 

%I learn how to make geometry: layer, station, super staion, envelope (hierachy).
%I define materials for our detector and CODEX-b geometry such as concrete, Herschel detector.
%Layer consists of silicon, station consists of aluminum. 
%There is a veto cone with two lead and one silicon. 
%Also just in front of CODEX-b, concrete wall exists to veto muons.
%First of all, using muon particle gun with high energy, test our geometry.
%And then using HepMC to generate pp collisions and do the same process as muon particle gun.
%I made hierachy system to build CODEX-b (envelop, super station, station, layer).
%I could check energy deposits and positions of CODEX-b hits.

\subsection{Simulation geometry}
First geometry is the CODEX-b.
CODEX-b consists of two parts face station and inner station.
Based on the paper, face station has 6 resistive plate chambers (RPCs) layers at 4~cm intervals with 1~cm granularity.
The size of each layer is $10 \times 10$~$m^{2}$ and the thickness is 2~cm. 
In this simulation we had been implemented layers as a tracker instead of RPCs.
Inner station also has same configuration except number of layers.
It will be equally spaced with triplets along the depth to minimize distance between reconstructed vertex and 1st measurement. 

%To make coincidence setup with test-bench, I made two Herschel plates with the same positions where all equipment have set.
%Two plates with 30 x 30~$cm^{2}$ size, 2~cm thickness.
%There is a concrete wall in front of scintillators.
%3~m thickness to suppress particles from pp collisions.
%Roughly in 1000 events, it has hits on scinitillators 4 - 9 events.
%There is a proposed veto cone. It consists of two lead absorbers and one silicon tracker.
%There is also concrete wall which blocks radiations (or particles) to reach CODEX-b box. 
%It has 3.2~m thickness.

\begin{figure}
\centering
\subfigure[]{
\centering
    \includegraphics[width=0.55\textwidth]{figs/INT/CODEXbBigGeo.pdf} 
}
\subfigure[]{
\centering
    \includegraphics[width=0.55\textwidth]{figs/INT/ZoomVersion.pdf}
}
\caption{
    CODEX-b simulation geometry: (a) overall, (b) close-up view. 
}
\end{figure}

We also created a concrete shield wall with 3.2~m thickness.
It was placed just front of CODEX-b box.
Between the LHCb box and the concrete wall, there is a proposed veto cone.
It contains two lead absorber and one active silicon layer.

Second geometry was consists of two scintillator plates which is the same as our measurement configurations.
The material of scintillator was  

\subsection{Simulation status}

We designed two different detectors based on the paper and the measurement, the former is the CODEX-b and the latter is a scintillator.
Both were tested with $\mu$ particle gun with 1\tev and the minimum bias events generated from the standalone Gauss. 

\begin{figure}
\centering
\subfigure[]{
\centering
    \includegraphics[width=0.9\textwidth]{figs/INT/Minbias.pdf}
}
\subfigure[]{
\centering
    \includegraphics[width=0.9\textwidth]{figs/INT/Scint.pdf}
}
\caption{
    Validation of the {\tt DD4Hep} based simulation with the concrete shield wall removed using minimum bias events: (a) CODEX-b box, and (b) two-plate scintillators for measurement campaign.}
\end{figure}

%To check hits on the layers of the CODEX-b, we removed the concrete wall. Fig 8 upper plot is shown the results.
There was no hits on the CODEX-b layers when tested minbias events with the concrete wall.
We decided to remove the shield wall to check performance of layers.
The Figure 8 upper plot is shown that hits from minbias events.
Also we recovered the concrete wall and changed CODEX-b geometry to two scintillator plates and tested with minbias events.
Following the lower plot of Fig 8, there is no hit on the scintillators.
Because its size is too small to measure hits and the concrete wall blocks particles from collisions.

%\begin{figure}[h]
%\centering
%\caption{
%    Test of the scintialltor configuration with minimum bias events
%}
%\end{figure}
